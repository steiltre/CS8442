%        File: hw4.tex
%     Created: Sun Apr 02 11:00 AM 2017 C
% Last Change: Sun Apr 02 11:00 AM 2017 C
%

\documentclass[11pt]{article}

\title{CSci 8442 Homework 4 }
\date{4/17/17}
\author{Trevor Steil}

\usepackage{amsmath}
\usepackage{amsthm}
\usepackage{amssymb}
\usepackage[margin=1.0in]{geometry}
\usepackage{esint}
\usepackage{enumitem}
\usepackage{algorithm}
\usepackage{algorithmicx}
\usepackage{algpseudocode}
\usepackage{bbm}
\usepackage{xcolor}

\newtheorem{theorem}{Theorem}[section]
\newtheorem{corollary}{Corollary}[section]
\newtheorem{proposition}{Proposition}[section]
\newtheorem{lemma}{Lemma}[section]
\newtheorem*{claim}{Claim}
%\newtheorem*{problem}{Problem}
%\newtheorem*{lemma}{Lemma}
\newtheorem{definition}{Definition}[section]

\newcommand{\R}{\mathbb{R}}
\newcommand{\N}{\mathbb{N}}
\newcommand{\C}{\mathbb{C}}
\newcommand{\Z}{\mathbb{Z}}
\newcommand{\Q}{\mathbb{Q}}
\newcommand{\E}{\mathbb{E}}
\newcommand{\supp}[1]{\mathop{\mathrm{supp}}\left(#1\right)}
\newcommand{\lip}[1]{\mathop{\mathrm{Lip}}\left(#1\right)}
\newcommand{\curl}{\mathrm{curl}}
\newcommand{\la}{\left \langle}
\newcommand{\ra}{\right \rangle}
\renewcommand{\vec}[1]{\mathbf{#1}}
\renewcommand{\div}{\mathrm{div}}

\newenvironment{problem}{\textbf{Problem.}}

\newenvironment{solution}[1][]{\emph{Solution #1}}

\algnewcommand{\Or}{\textbf{ or }}
\algnewcommand{\And}{\textbf{ or }}

\begin{document}
\maketitle

\begin{enumerate}
  \item
    \begin{problem}
      An important special case of the point location problem is when the underlying subdivision is a simple polygon: the answer to a point location
      query here is whether a query point lies inside or outside the polygon. In this case, the query can be answered efficiently using a simpler data
      structure than the one used for general subdivisions.

      Let $P$ be a $n$-vertex simple polygon, with its vertices given in clockwise order in an array. For each of the following cases, describe
      briefly but precisely, in words, how to answer a point location query for a query point $q$ in time $O(\log n)$ using $O(n)$ space. (Pseudocode
      is not needed.) Pre-processing of $P$ is allowed. State any data structure you use.

      \begin{enumerate}
        \item $P$ is convex
        \item $P$ is $y$-monotone
        \item $P$ is star-shaped, i.e., there is a point, $r$, inside $P$ from which all points, $s$, inside $P$ are visible along a straight line,
          i.e., the line segment $\overline{rs}$ is contained in $P$. Assume that you are given $r$.
      \end{enumerate}

    \end{problem}

    \begin{solution}
      \begin{enumerate}
        \item
          Let a horizontal line intersect $P$ in two points $p$ and $q$. Then by definition of convexity, all points between $p$ and $q$ must also
          belong to $P$. Therefore, any horizontal line can intersect $P$ at most twice, and $P$ is $y$-monotone. Therefore, we will use the algorithm
          given in part (b) for convex polygons as well.

        \item
          We begin by sorting the points of $P$ by $y$-coordinate. Call the points sorted in this way $p_1, \dots, p_n$. Consider horizontal lines
          passing between $p_i$ and $p_{i+1}$ for some $1 \leq i \leq n-1$. There are no points of $P$ in the slab between these two lines because of
          the sorting. Because $P$ is $y$-monotone, the horizontal lines intersect $P$ in at most two points each. Because of these facts, we have a
          unique segment defining the right side of $P$ in this slab and a unique segment defining the left side of $P$ in this slab.

          Guided by the observations above, we place the points of $P$ in an array $A$ sorted by nondecreasing $y$-coordinate. Associated to each adjacent
          pair in $A$, we store the segments defining the left and right segments of $P$. To query our array with a point $q$, we perform a binary search to determine
          the pair of points $p_i$, $p_{i+1}$ that define the horizontal slab $q$ is in. If $q$ is found to lie below $p_1$ or above $p_n$, it cannot
          be inside $P$. Otherwise, we use sidedness tests to determine if $q$ is in $P$. If $q$ is to the right of the associated left segment of $P$
          and to the left of the associated right segment of $P$, then the query returns ``true''. Otherwise, the query returns ``false''. In the case
          that $q$ has the same $y$-coordinate as a point $p$ defining $P$, then either slab above or below $p$ can be used for the query (unless $p$
          is the highest or lowest point in $P$, in which case the slab intersecting the interior of $P$ must be used).

          Correctness of this query should be clear. A query point is contained in $P$ if and only if it intersects one of the horizontal slabs
          defined by the points sorted by nondecreasing $y$-coordinate and is found between the left and right segments of the boundary of $P$ that
          intersect this slab. These conditions are exactly the conditions checked above.

          This algorithm sorts points of $P$, giving a precomputing time of $O(n \log n)$. $A$ stores each of the $n$ vertices of $P$ exactly once,
          and associates $O(1)$ edges to each adjacent pair in $A$. Therefore, the storage requirements are $O(n)$. Querying requires a binary search
          on $A$ followed by a constant number of sidedness tests, giving a query time of $O(\log n)$.

        \item
          For star-shaped polygons, the algorithm looks very similar to that of $y$-monotone polygons. Without loss of generality, we may assume $r$
          is placed at the origin by shifting $P$. We want the points defining $P$ to be sorted in angular order relative to the positive $x$-axis.
          The points are given in clockwise order, so this is already the case. If we needed to sort them, we could use the sorting algorithm from
          Homework 1.

          As in part (b), we have an array $A$ with the points defining $P$ sorted, this time in angular order. To query this structure with a query
          point $q$, we perform a binary search using the constant time angle comparison operator. This returns a pair of points $p_i$ and $p_{i+1}$
          such that $q$ is intersected by a ray from the origin as it sweeps clockwise between $p_i$ and $p_{i+1}$. Next, we determine if the segments
          $\overline{rq}$ and $\overline{p_i p_{i+1}}$ intersect. If they do, $q$ is outide $P$. If they don't, $q$ is inside $P$. In this radial
          problem, we consider the array $A$ having wraparound at its ends so $p_1$ and $p_n$ are adjacent. If $q$ induces the same angle with the
          positive $x$-axis as a vertex $p_i$ of $P$, $p_{i-1}$ or $p_{i+1}$ can be used as the second point defining the segment on the boundary of
          $P$. The test for segment intersection could be replaced by a sidedness test, but the segment intersection is more indicative of what is
          happening with the definition of star-shaped polygons.

          The precomputing time is $O(n \log n)$ if the points needed to be placed in angular order. If they are already in order, no precomputing is
          needed. The array $A$ stores only the vertices of $P$, so the storage required is $O(n)$. The query algorithm requires a binary search and a
          single segment intersection test, so the query time is $O(\log n)$.

      \end{enumerate}
    \end{solution}

  \item
    \begin{problem}
      Let $S = \{p_1, p_2,\dots, p_n\}$ be a set of $n$ sites (points) in the plane. Let $Vor(S)$ be the Voronoi Diagram of $S$. Let $B$ be an
      axes-parallel bounding box that is large enough to enclose all the vertices of $Vor(S)$. Assume that the subdivision consisting of $B$ and the
      part of $Vor(S)$ that is inside $B$ has been computed into a DCEL structure. (See the figure on p. 166 for an example of such a subdivision.)
      Assume that each (finite) face of the DCEL also stores the associated site.

      A site $p_j$ is a nearest neighbor of another site $p_i$ if the Euclidean distance between $p_i$ and $p_j$ is no larger than the Euclidean
      distance between $p_i$ and any other site. (Note that $p_i$ can have more than one nearest neighbor. Also note that if $p_j$ is the nearest
      neightbor of $p_i$, then it is not necessary that $p_i$ is the nearest neight of $p_j$.) Computing efficiently the nearest neighbors of each
      site in $S$ is an important problem, with application in clustering, collision avoidance, etc.

      \begin{enumerate}
        \item Let $V(p_i)$ be the Voronoi polygon of $p_i$. Prove that every nearest neighbor of $p_i$ defines an edge of $V(p_i)$. (Note that the
          converse is not necessarily true, i.e. every edge of $V(p_i)$ need not define a nearest neighbor of $p_i$.)

        \item Show how to use $Vor(S)$ and the result in part (a) to compute for each site $p_i \in S$ all of its nearest neighbors. (This implies a
          solution to the closest pair problem on $S$.) The overall running time should be $O(n)$. You may describe your solution in words, but
          please be clear and precise.
      \end{enumerate}
    \end{problem}

    \begin{solution}
       \begin{enumerate}
         \item
          The nearest neighbors of $p_i$ all lie on a circle $C$ that is centered at $p_i$ and contains no sites in its interior (other than $p_i$).
          If we take a site $p_j$ on $C$, we can consider a circle $C'$ obtained by shrinking $C$ to have $p_i$ and $p_j$ on its boundary. Then $C'
          \subset C$ because $C$ and $C'$ are tangent at $p_j$ and $C'$ has a smaller radius than $C$. Therefore, the interior of $C'$ contains no
          sites of $S$. Also, $C'$ has no additional sites on its boundary because the only point it shares with $C$ is $p_j$. Therefore, $C'$ is a
          circle with exactly two sites on its boundary and no sites in its interior. Thus we have a Voronoi edge between $p$ and $q$ in $Vor(S)$.
          This Voronoi edge is an edge of $V(p_i)$.

         \item
           We need to be able to take $Vor(S)$ and ``remove'' edges that do not correspond to a nearest neighbor (we will not actually remove any
           edges from the DCEL). To do this, we can walk along each face of the DCEL. Each face is associated to a site $p_i \in S$. Any nearest
           neighbor to $p_i$ will share a Voronoi edge with $p_i$ by part (a). Because of this, we can walk along the edges bounding $V(p_i)$ and
           compute the distance to the site associated to each edge. The sites that have the minimum distance will be the nearest neighbors for $p_i$.
           By repeating this for all vertices, we get the nearest neighbors for every vertex in $S$.

           In this algorithm, we walk along each of the edges of the Voronoi diagram twice. At each edge, we do a constant amount of work. We know the
           number of edges in a Voronoi diagram is $O(n)$, so we have a $O(n)$ algorithm. If we want to be slightly more careful and consider the
           effects of $B$, we see that $B$ has four sides, so no face of $Vor(S)$ can share more than four edges with $B$. The number of faces is
           $O(n)$, so we get another $O(n)$ contribution from walking around the edges of $B$.
       \end{enumerate}

     \end{solution}

   \item
     \begin{problem} Ex 9.13, p. 217

       The \emph{Gabriel graph} of a set $P$ of points in the plane is defined as follows: Two points $p$ and $q$ are connected by an edge of the
       Gabriel graph if and only if the disc with diameter $pq$ does not contain any other point of $P$.

       \begin{enumerate}
         \item Prove that $DG(P)$ contains the Gabriel graph of $P$.

         \item
           Prove that $p$ and $q$ are adjacent in the Gabriel graph of $P$ if and only if the Delaunay edge between $p$ and $q$ intersects its dual
           Voronoi edge.

         \item Give an $O(n \log n)$ time algorithm to compute the Gabriel graph of a set of $n$ points.

       \end{enumerate}

     \end{problem}

     \begin{solution}

       \begin{enumerate}
         \item
           Take any edge $e$ in the Gabriel graph. Let $p$ and $q$ be the endpoints of $e$. Then the disc with diameter $pq$ does not contain any
           other point of $P$. Therefore, there is an edge between $p$ and $q$ in the Voronoi diagram of $P$, which implies there is an edge between
           $p$ and $q$ in the Delaunay graph. This holds for any edge in the Gabriel graph, so the Gabriel graph is contained in $DG(P)$.

         \item
           Assume the Delaunay edge between $p$ and $q$ intersects the dual Voronoi edge. Let this intersection point be $r$. Then $r$ is on the
           boundary between $Vor(p)$ and $Vor(q)$. One of the structural properties of Voronoi diagrams listed in class says there is a circle with
           $p$ and $q$ on its boundary with no other sites of $P$ on the boundary or in the interior. $r$ is on the (straight) Delaunay edge between
           $p$ and $q$, so this circle has $pq$ as its diameter. Therefore, $p$ and $q$ are adjacent in the Gabriel graph of $P$.

           Now, assume $p$ and $q$ are adjacent on the Gabriel graph. Then we can draw the disc with diameter $pq$. Let $r$ be the midpoint of the
           line between $p$ and $q$, i.e. $r$ is the center of the disc. The disc contains no sites other than $p$ and $q$ by the definition of the
           Gabriel graph. Therefore, $r$ is on the Voronoi edge between $p$ and $q$. This means $r$ is on the Voronoi edge and the line segment
           between $p$ and $q$, but this line segment is exactly the Delaunay edge between $p$ and $q$. Thus the Voronoi edge and Delaunay edge
           intersect.

         \item
           To construct a Gabriel graph, we can begin by computing the Delaunay graph and the Voronoi diagram. For every edge in the Delaunay graph,
           we take the associated edge in the Voronoi diagram and check if the Voronoi edge and Delaunay edge intersect. If they do, we leave the edge
           in the graph. Otherwise, we remove it. By (a) and (b), this algorithm will return the correct graph.

           Computing the Voronoi diagram and Delaunay graph take $O(n \log n)$ time. We know there are $O(n)$ edges in each, so we can use our
           constant time intersection check to reduce to the Gabriel graph in $O(n)$ time. Combining these steps, we have a run-time of $O(n \log n)$.

       \end{enumerate}

     \end{solution}
\end{enumerate}
\end{document}


