%        File: hw2.tex
%     Created: Thu Feb 16 02:00 PM 2017 C
% Last Change: Thu Feb 16 02:00 PM 2017 C
%

\documentclass[a4paper]{article}

\title{CSci 8442 Homework 2 }
\date{3/6/17}
\author{Trevor Steil}

\usepackage{amsmath}
\usepackage{amsthm}
\usepackage{amssymb}
\usepackage{esint}
\usepackage{enumitem}
\usepackage{algorithm}
\usepackage{algorithmicx}
\usepackage{algpseudocode}
\usepackage{bbm}
\usepackage{xcolor}

\newtheorem{theorem}{Theorem}[section]
\newtheorem{corollary}{Corollary}[section]
\newtheorem{proposition}{Proposition}[section]
\newtheorem{lemma}{Lemma}[section]
\newtheorem*{claim}{Claim}
%\newtheorem*{problem}{Problem}
%\newtheorem*{lemma}{Lemma}
\newtheorem{definition}{Definition}[section]

\newcommand{\R}{\mathbb{R}}
\newcommand{\N}{\mathbb{N}}
\newcommand{\C}{\mathbb{C}}
\newcommand{\Z}{\mathbb{Z}}
\newcommand{\Q}{\mathbb{Q}}
\newcommand{\E}{\mathbb{E}}
\newcommand{\supp}[1]{\mathop{\mathrm{supp}}\left(#1\right)}
\newcommand{\lip}[1]{\mathop{\mathrm{Lip}}\left(#1\right)}
\newcommand{\curl}{\mathrm{curl}}
\newcommand{\la}{\left \langle}
\newcommand{\ra}{\right \rangle}
\renewcommand{\vec}[1]{\mathbf{#1}}
\renewcommand{\div}{\mathrm{div}}

\newenvironment{problem}{\textbf{Problem.}}

\newenvironment{solution}[1][]{\emph{Solution #1}}

\algnewcommand{\Or}{\textbf{ or }}
\algnewcommand{\And}{\textbf{ or }}

\begin{document}
\maketitle

\begin{enumerate}
  \item
    \begin{problem}
      Let $P$ be any polygon, possibly with holes.
      \begin{enumerate}
        \item Prove that there always exists a triangulation of $P$. (Consider reducing $P$ to a polygon with one fewer hole and applying induction.)
        \item Derive exact expressions for the number of triangles and the number of diagonals in any triangulation of $P$, as a function of the
          number, $n$, of vertices and the number, $h$, of holes in $P$. (Consider making an educated guess for the expressions and verifying these by
          induction on $h$.)
      \end{enumerate}
    \end{problem}

    \begin{solution}
      \begin{enumerate}
        \item
          We have the existence of a triangulation by noting that the discussion of splitting a polygon into $y$-monotone pieces holds whether $P$ has
          holes or not. Merge and split vertices are defined locally, so the definition holds for vertices on the boundary of $P$ as well as holes
          inside of $P$.

          The same algorithm as in the case without holes can be used to remove merge and split vertices by adding valid diagonals. As before, this means the original polygon
          has been split into $y$-monotone pieces. Each of these $y$-monotone pieces cannot contain a hole because this would create a $y$-value where
          a horizontal line would intersect in more than a single segment. Once we have the $y$-monotone pieces, we can triangulate in the same way.

        \item
      \end{enumerate}
    \end{solution}

  \item 5.13 page 120 \\
    \begin{problem}
      In many applications one wants to do range searching among objects other than points.
      \begin{enumerate}
        \item Let $S$ be a set of $n$ axis-parallel rectangles in the plane. We want to be able to report all rectangles in $S$ that are completely
          contained in a query rectangle $[x:x'] \times [y:y']$. Describe a data structure for this problem that uses $O(n \log^3 n)$ storage and has
          ${O(\log^4 n + k)}$ query time, where $k$ is the number of reported answers. Hint: Transform the problem to an orthogonal range searching
          problem in some higher-dimensional space.

        \item
          Let $P$ consist of a set of $n$ polygons in the plane. Again describe a data structure that uses $O(n \log^3 n)$ storage and has ${O(\log^4 n
          + k)}$ query time to report all polygons completely contained in the query rectangle, where $k$ is the number of reported answers.

        \item
          Improve the query time of your solutions (both a and b) to ${O(\log^3 n + k)}$.
      \end{enumerate}
    \end{problem}

    \begin{solution}
      \begin{enumerate}
        \item

        \item

        \item
      \end{enumerate}
    \end{solution}

  \item 5.5 page 118 parts (a) and (b) \\
    \begin{problem}
      Algorithm \texttt{SearchKDTree} can also be used when querying with other ranges than rectangles. For example, a query is answered correctly if
      the range is a triangle.
      \begin{enumerate}
        \item
          Show that the query time for range queries with triangles is linear in the worst case, even if no answers are reported at all. Hint: Choose
          all points to be stored in the kd-tree on the line $y=x$.

        \item
          Suppose that a data structure is needed that can answer triangular range queries, but only for triangles whose edges are horizontal,
          vertical, or have slope $+1$ or $-1$. Develop a linear size data structure that answers such range queries in $O(n^{3/4} + k)$ time, where
          $k$ is the number of points reported. Hint: Choose 4 coordinate axes in the plane and use a 4-dimensional kd-tree.
      \end{enumerate}
    \end{problem}

    \begin{solution}
      \begin{enumerate}
        \item

        \item
      \end{enumerate}
    \end{solution}
\end{enumerate}<++>
\end{document}


