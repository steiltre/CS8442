%        File: hw2.tex
%     Created: Thu Feb 16 02:00 PM 2017 C
% Last Change: Thu Feb 16 02:00 PM 2017 C
%

\documentclass[a4paper]{article}

\title{CSci 8442 Homework 2 }
\date{3/6/17}
\author{Trevor Steil}

\usepackage{amsmath}
\usepackage{amsthm}
\usepackage{amssymb}
\usepackage{esint}
\usepackage{enumitem}
\usepackage{algorithm}
\usepackage{algorithmicx}
\usepackage{algpseudocode}
\usepackage{bbm}
\usepackage{xcolor}

\newtheorem{theorem}{Theorem}[section]
\newtheorem{corollary}{Corollary}[section]
\newtheorem{proposition}{Proposition}[section]
\newtheorem{lemma}{Lemma}[section]
\newtheorem*{claim}{Claim}
%\newtheorem*{problem}{Problem}
%\newtheorem*{lemma}{Lemma}
\newtheorem{definition}{Definition}[section]

\newcommand{\R}{\mathbb{R}}
\newcommand{\N}{\mathbb{N}}
\newcommand{\C}{\mathbb{C}}
\newcommand{\Z}{\mathbb{Z}}
\newcommand{\Q}{\mathbb{Q}}
\newcommand{\E}{\mathbb{E}}
\newcommand{\supp}[1]{\mathop{\mathrm{supp}}\left(#1\right)}
\newcommand{\lip}[1]{\mathop{\mathrm{Lip}}\left(#1\right)}
\newcommand{\curl}{\mathrm{curl}}
\newcommand{\la}{\left \langle}
\newcommand{\ra}{\right \rangle}
\renewcommand{\vec}[1]{\mathbf{#1}}
\renewcommand{\div}{\mathrm{div}}

\newenvironment{problem}{\textbf{Problem.}}

\newenvironment{solution}[1][]{\emph{Solution #1}}

\algnewcommand{\Or}{\textbf{ or }}
\algnewcommand{\And}{\textbf{ or }}

\begin{document}
\maketitle

\begin{enumerate}
  \item
    \begin{problem}
      Let $P$ be any polygon, possibly with holes.
      \begin{enumerate}
        \item Prove that there always exists a triangulation of $P$. (Consider reducing $P$ to a polygon with one fewer hole and applying induction.)
        \item Derive exact expressions for the number of triangles and the number of diagonals in any triangulation of $P$, as a function of the
          number, $n$, of vertices and the number, $h$, of holes in $P$. (Consider making an educated guess for the expressions and verifying these by
          induction on $h$.)
      \end{enumerate}
    \end{problem}

    \begin{solution}
      \begin{enumerate}
        \item
          We have the existence of a triangulation by noting that the discussion of splitting a polygon into $y$-monotone pieces holds whether $P$ has
          holes or not. Merge and split vertices are defined locally, so the definition holds for vertices on the boundary of $P$ as well as holes
          inside of $P$.

          The same algorithm as in the case without holes can be used to remove merge and split vertices by adding valid diagonals. As before, this means the original polygon
          has been split into $y$-monotone pieces. Each of these $y$-monotone pieces cannot contain a hole because this would create a $y$-value where
          a horizontal line would intersect in more than a single segment. Once we have the $y$-monotone pieces, we can triangulate in the same way.

        \item
          Consider a fully triangulated polygon with $h$ holes and $n$ vertices. Each face is one of the following: a triangle, a hole, or the
          exterior of the polygon. Therefore, there are $t + h + 1$ faces, where $t$ is the number of triangles in the triangulation. A simple polygon
          has the same number of edges as vertices as edges. Our original polygon is composed of simple polygons, so the number of edges bounding the
          exterior of the polygon and the interior of the holes is $n$. In the triangulated polygon, every edge is a diagonal or an edge in the
          original polygon. Therefore, we have $n + d$ egdes, where $d$ is the number of diagonals in the triangulation.

          Substituting these values into Euler's formula, we get
          \begin{align*}
            2 &= \# \text{faces} - \# \text{edges} + \# \text{vertices} \\
            &= (t + h + 1) - (n + d) + n \\
            &= t + h - d + 1
          \end{align*}
          This simplifies to
          \begin{equation}\label{eq:Euler_app}
            t + h - d = 1
          \end{equation}

          Now, consider walking around the edges of each triangle in the triangulation. Each triangle has 3 edges. Looking at the edges that are
          traversed in this way, we walk along each diagonal exactly twice and each edge of $P$ exactly once. This gives
          \[ 3t = 2d + n .\]

          Multiplying \eqref{eq:Euler_app} by 3 and inserting this relation, we have
          \begin{align*}
            3 &= 3t + 3h - 3d \\
            &= 3h + n - d
          \end{align*}
          This simplifies to $d = n - 3 + 3h$.

          Multiplying \eqref{eq:Euler_app} by 2 and again using the relation, we have
          \begin{align*}
            2 &= 2t + 2h - 2d \\
            &= 2t + 2h  - (3t - n) \\
            &= -t + 2h + n
          \end{align*}
          This simplifies to $t = n - 2 + 2h$.
      \end{enumerate}
    \end{solution}

  \item 5.13 page 120 \\
    \begin{problem}
      In many applications one wants to do range searching among objects other than points.
      \begin{enumerate}
        \item Let $S$ be a set of $n$ axis-parallel rectangles in the plane. We want to be able to report all rectangles in $S$ that are completely
          contained in a query rectangle $[x:x'] \times [y:y']$. Describe a data structure for this problem that uses $O(n \log^3 n)$ storage and has
          ${O(\log^4 n + k)}$ query time, where $k$ is the number of reported answers. Hint: Transform the problem to an orthogonal range searching
          problem in some higher-dimensional space.

        \item
          Let $P$ consist of a set of $n$ polygons in the plane. Again describe a data structure that uses $O(n \log^3 n)$ storage and has ${O(\log^4 n
          + k)}$ query time to report all polygons completely contained in the query rectangle, where $k$ is the number of reported answers.

        \item
          Improve the query time of your solutions (both a and b) to ${O(\log^3 n + k)}$.
      \end{enumerate}
    \end{problem}

    \begin{solution}
      \begin{enumerate}
        \item
          This query problem can be transformed into a query problem in 4-dimensions by recognizing we have 4 coordinates we are interested in for
          each rectangle in $S$: the $x$-coordinate of the left edge, the $x$-coordinate of the right edge, the $y$-coordinate of the bottom edge, and
          the $y$-coordinate of the top edge. Using this we can encode each rectangle using the coordinates $(x_l, x_r, y_b, y_t)$, where the
          coordinates are those listed previously.

          For querying against these points in $\R^4$, we query using the $x$-interval of our query rectangle twice in a row (using the same values
          each time), followed by two queries using the $y$-interval of our query rectangle. This query process returns a rectangle if and only if
          both $x$-coordinates and both $y$-coordinates of the point in $\R^4$ lie within the query rectangle (with two copies of its query
          intervals), that is, if and only if all edges are contained in the query rectangle.

          By using a range tree, we get the desired $O(n \log^3 n)$ storage and $O( \log^4 n + k)$ query time.

        \item
          This algorithm works the same way as that of part (a). Now we store each polygon in $P$ as $(x_l, x_r, y_b, y_t)$, where $x_l$ and $x_r$ are
          the left- and right-most $x$-coordinates of the polygon, and $y_b$ and $y_t$ are the smallest and largest $y$-coordinates of points in the
          polygon. By querying twice in each direction, we find all polygons in $P$ where all points have $x$- and $y$-coordinates in the ranges
          defined by the query rectangle.

          By using a range tree, we get the desired $O(n \log^3 n)$ storage and $O( \log^4 n + k)$ query time.

        \item Use fractional cascading on the $\R^2$ range trees constructed as part of the $\R^4$ range trees being used.
      \end{enumerate}
    \end{solution}

  \item 5.5 page 118 parts (a) and (b) \\
    \begin{problem}
      Algorithm \texttt{SearchKDTree} can also be used when querying with other ranges than rectangles. For example, a query is answered correctly if
      the range is a triangle.
      \begin{enumerate}
        \item
          Show that the query time for range queries with triangles is linear in the worst case, even if no answers are reported at all. Hint: Choose
          all points to be stored in the kd-tree on the line $y=x$.

        \item
          Suppose that a data structure is needed that can answer triangular range queries, but only for triangles whose edges are horizontal,
          vertical, or have slope $+1$ or $-1$. Develop a linear size data structure that answers such range queries in $O(n^{3/4} + k)$ time, where
          $k$ is the number of points reported. Hint: Choose 4 coordinate axes in the plane and use a 4-dimensional kd-tree.
      \end{enumerate}
    \end{problem}

    \begin{solution}
      \begin{enumerate}
        \item
          Consider the set of points $P = \{(k,k) \ | \ k = 1,\dots,n\}$ and a sufficiently large query triangle bounded above by the edge given by the line $y = x -
          \frac{1}{2}$. This triangle is going to intersect $O(n)$ regions in the associated kd-tree.

          Consider any vertical splitter $l$ of the kd-tree. This line passes through some point $(k,k) \in P$. Consider any region with $l$ as its
          left boundary. If this region has not bottom right corner, it is unbounded in the positive $x$ or negative $y$ direction (or both). Either way, it
          will contain points with $y < x - \frac{1}{2}$ and will therefore intersect the query triangle. Now we must understand when this region has
          a bottom-right corner.

          The right edge of this region is a vertical line that passes through a point $(i,i)$ for some $i$. The bottom edge is a horizontal segment that
          passes through a point $(j,j)$. Because this horizontal segment intersects the left and right boundaries, we have $k \leq j \leq i$. Also,
          we claim $i \neq j$. As the kd-tree is constructed, regions are split by choosing vertical and horizontal segments that pass through medians
          of points ordered in the $x$- or $y$-directions. After a splitting, the median value will be to the right of or above all points it is
          compared against for future splittings because all points lie on the line $y=x$. This means the same point will not be chosen as another
          median. Therefore, $k < j < i$.

          Because $j < i$, the lower-right corner of the region being considered is at $(i,j)$ with $j \leq i-1$. Therefore, $(i,j)$ is in the query
          rectangle. Therefore, every region with a left edge intersects the query triangle. Similarly, we can argue any region with a right edge
          intersects the query triangle. It is obvious that every region with no left or right edges must intersect the query triangle. Thus, every
          region in the kd-tree intersects the query triangle.

          Because every region of the kd-tree intersects the query triangle, every leaf of the tree will be explicitly checked for containment, giving
          $O(n)$ as the lower bound for querying. In this example, it is easy to see no points of $P$ were actually contained in the query rectangle.

        \item
          To construct the data structure, we perform the same basic steps as for a kd-tree except splitters will rotate between vertical lines,
          horizontal lines, lines with slope 1, and lines with slope -1. To do the splitting with the oblique lines, you can consider rotating the
          plane 45 degrees counterclockwise and performing vertical and horizontal splittings in these rotated coordinates.

          To query the structure, the same \texttt{SearchKDTree} algorithm can be used. This algorithm descends from the root, searching branches that
          correspond to regions the query triangle intersects with. When a region of any node in the tree is completely contained in the query
          triangle, the entire subtree is reported.

          As before, each point is contained in a single leaf node. The total number of nodes in the tree is $2n-1$, so the storage cost is $O(n)$.

          The query time can be established in the same way as for a kd-tree. The main difference is that we no longer need to descend two levels in
          the tree to establish the recursion, but must descend four levels.

          First, we see that the time required to traverse the subtrees which are reported is $O(k)$. The other contribution to the query time is the
          number of nodes that must be visited but are not in a reported subtree. Consider an infinite vertical line $l$. Let $Q(n)$ be the number of
          regions in our modified kd-tree storing $n$ points with a vertical splitter at the root that $l$ intersects.

          At the root of this tree, $l$ intersects the region (i.e. $\R^2$). Descending one level, $l$ intersects exactly one of the two regions split
          by a vertical line. At the next level, $l$ intersects all regions that are children of intersected regions above because the regions are
          split by a horizontal line. Descending one more level, there are four children that $l$ can possibly intersect. Given the regions are split
          by an oblique line at this level, there can be between two and four intersections. At the next level, we have nodes with vertical splitters
          again. We again are splitting with an oblique line, which we cannot say for certain reduces the number of regions we must consider.
          Therefore, we have to consider up to eight of the total sixteen nodes at this level because the original vertical split was the only one
          that definitely removed nodes from consideration.

          By descending four levels as above, $l$ has intersected at most eight regions. The nodes at the next level with vertical splitters are roots
          of trees that contain $n/16$ points (from descending four levels, halving the number of points each time). This gives the recursion
          \[ Q(n) \leq
            \begin{cases}
              O(1) &\text{if } n = 1 \\
              O(1) + 8 Q \left( \frac{n}{16} \right) &\text{if } n > 1
            \end{cases}
          \]

          Applying the Master Theorem to this recursion, we get $Q(n) = O(n^{3/4})$. By the same argument, we can bound the number of regions
          intersected by horizontal and oblique query lines. Therefore, the total query time is $O(n^{3/4} + k)$.

      \end{enumerate}
    \end{solution}
\end{enumerate}
\end{document}


