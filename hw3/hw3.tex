%        File: hw3.tex
%     Created: Fri Mar 10 01:00 PM 2017 C
% Last Change: Fri Mar 10 01:00 PM 2017 C
%

\documentclass[11pt]{article}

\title{CSci 8442 Homework 3}
\date{3/27/17}
\author{Trevor Steil}

\usepackage{amsmath}
\usepackage{amsthm}
\usepackage{amssymb}
\usepackage{esint}
\usepackage{enumitem}
\usepackage{algorithm}
\usepackage{algorithmicx}
\usepackage{algpseudocode}
\usepackage{bbm}
\usepackage{xcolor}

\newtheorem{theorem}{Theorem}[section]
\newtheorem{corollary}{Corollary}[section]
\newtheorem{proposition}{Proposition}[section]
\newtheorem{lemma}{Lemma}[section]
\newtheorem*{claim}{Claim}
%\newtheorem*{problem}{Problem}
%\newtheorem*{lemma}{Lemma}
\newtheorem{definition}{Definition}[section]

\newcommand{\R}{\mathbb{R}}
\newcommand{\N}{\mathbb{N}}
\newcommand{\C}{\mathbb{C}}
\newcommand{\Z}{\mathbb{Z}}
\newcommand{\Q}{\mathbb{Q}}
\newcommand{\E}{\mathbb{E}}
\newcommand{\supp}[1]{\mathop{\mathrm{supp}}\left(#1\right)}
\newcommand{\lip}[1]{\mathop{\mathrm{Lip}}\left(#1\right)}
\newcommand{\curl}{\mathrm{curl}}
\newcommand{\la}{\left \langle}
\newcommand{\ra}{\right \rangle}
\renewcommand{\vec}[1]{\mathbf{#1}}
\renewcommand{\div}{\mathrm{div}}

\newenvironment{problem}{\textbf{Problem.}}

\newenvironment{solution}[1][]{\emph{Solution #1}}

\algnewcommand{\Or}{\textbf{ or }}
\algnewcommand{\And}{\textbf{ or }}

\begin{document}
\maketitle

\begin{enumerate}
  \item 10.8 pg. 420

    \begin{problem}
        Segment trees can be used for multi-level data structures.
        \begin{enumerate}
          \item Let $R$ be a set of $n$ axis-parallel rectangles in the plane. Design a data structure for $R$ such that the rectangles containing a query
            point $q$ can be reported efficiently. Analyze the amount of storage and the query time of your data structure. \textit{Hint:} Use a segment
            tree on the $x$-intervals of the rectangles, and store canonical subsets of the nodes in this segment tree in an appropriate associated
            structure.

          \item Generalize this data structure to $dd$-dimensional space. Here we are given a set of axis-parallel hyperrectangles -- that is, polytopes
            of the form $[x_1 : x_1'] \times [x_2 : x_2' \time \dots \time [x_d : x_d']$ -- and we want to report the hyperrectangles containing a query
            point. Analyze the amount of storage and the query time of your data structure.
        \end{enumerate}

        A carefule description of the data structure, in words, suffices; pseudocode is not needed. (If it helps, augment your discussion with
        appropriate figures.) Be sure to justify the correctness of your method and to analyze its performance (space and query time).
    \end{problem}

    \begin{solution}
      \begin{enumerate}
        \item

        \item
      \end{enumerate}
    \end{solution}

  \item
    \begin{problem}

      Let $S$ be set of $n$ points in the plane. In class we discussed how to preprocess $S$ in $O(n \log n)$ time into a layered (i.e.,
      fractional-cascaded) 2D range tree of size $O(n \log n)$ so that the $k$ points that are contained in a query rectangle $q = [x_1, x_x] \times
      [y_1, y_2]$ can be reported in $O(\log n + k)$ time.

      It is possible to achieve the same bounds using a (relatively) sipler data structure: a 1D range tree where each node stores a suitable
      associated structure (different from a 1D range tree). Describe how to build and query the overall data structure and analyze its performance
      (space and query time). You may describe the construction and query algorithms in words, but please be clear and precise. (Again, augment your
      discussion with appropriate figures if needed.)

    \end{problem}

    \begin{solution}

    \end{solution}

  \item
    \begin{problem}
      This question pertains to the paper \textit{``Compuataional Geometry: Generalized (or colored) intersection searching''}, by P. Gupta, S. Rahul,
      M. Smid, and R. Janardan, which can be downloaded from the class web page. The paper is a recent survey of a general class of query-retrieval
      problems that has attracted widespread interest in recent years.

      The paper describes several techniques that have been designed over the years to address such problems. Please read the paper carefully and
      describe any two of the methods in detail. \textit{ You answer should be in your own words and should show that you have really understood the
      methods; do not just recite from the paper.} You are welcom to refer to any of the papers cited in the survey, if needed.

      P.S. You may find it fun to think on your own of other new problems that can be formulated in this setting.
    \end{problem}

    \begin{solution}

    \end{solution}
\end{enumerate}

\end{document}


