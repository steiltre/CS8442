%        File: project.tex
%     Created: Sun Feb 12 08:00 PM 2017 C
% Last Change: Sun Feb 12 08:00 PM 2017 C
%

\documentclass[a4paper]{article}

\title{Power Diagrams and Additively Weighted Voronoi Diagrams Proposal}
\date{2/20/17}
\author{Trevor Steil}

\usepackage{amsmath}
\usepackage{amsthm}
\usepackage{amssymb}
\usepackage[backend=biber,style=numeric,citestyle=numeric,style=numeric]{biblatex}
\usepackage{esint}
\usepackage{enumitem}
\usepackage{algorithm}
\usepackage{algorithmicx}
\usepackage{algpseudocode}
\usepackage{bbm}
\usepackage{xcolor}

\newtheorem{theorem}{Theorem}[section]
\newtheorem{corollary}{Corollary}[section]
\newtheorem{proposition}{Proposition}[section]
\newtheorem{lemma}{Lemma}[section]
\newtheorem*{claim}{Claim}
\newtheorem*{problem}{Problem}
%\newtheorem*{lemma}{Lemma}
\newtheorem{definition}{Definition}[section]

\newcommand{\R}{\mathbb{R}}
\newcommand{\N}{\mathbb{N}}
\newcommand{\C}{\mathbb{C}}
\newcommand{\Z}{\mathbb{Z}}
\newcommand{\Q}{\mathbb{Q}}
\newcommand{\E}{\mathbb{E}}
\newcommand{\supp}[1]{\mathop{\mathrm{supp}}\left(#1\right)}
\newcommand{\lip}[1]{\mathop{\mathrm{Lip}}\left(#1\right)}
\newcommand{\curl}{\mathrm{curl}}
\newcommand{\la}{\left \langle}
\newcommand{\ra}{\right \rangle}
\renewcommand{\vec}[1]{\mathbf{#1}}
\renewcommand{\div}{\mathrm{div}}

\algnewcommand{\Or}{\textbf{ or }}
\algnewcommand{\And}{\textbf{ or }}

\addbibresource{project.bib}

\begin{document}
\maketitle

\section{Introduction}

The closest-neighbors Voronoi diagram for a set of points $S \subset \R^2$ partitions the plane into regions where each region is associated to a
point $p \in S$ and is defined by
\begin{equation*}
  R_p = \{ q \in \R^2 | d(p,q) \leq d(p',q) \text{ for all } p' \in S \} .
\end{equation*}
where $d(p,q)$ is the Euclidean distance between $p$ and $q$.

There are several ways to generalize this concept to give other interesting partitions of the plane. Partitioning into regions defined by sharing the
same $k$ closest points in $S$ gives rise to an order-$k$ Voronoi diagram \cite{aurenhammer_survey}. The method for producing variations on
Voronoi diagrams we will consider is using weighted distance functions. Let $w: S \to \R^+$ be a function that defines the weight of points of $S$. The
(additively) weighted Voronoi diagram partitions the plane into regions defined by
\begin{equation*}
  \tilde{R}_p = \{ q \in \R^2 | d(p,q) - w(p) \leq d(p',q) - w(p') \text{ for all } p' \in S \}.
\end{equation*}
It is worth noting that $p$ is not necessarily contained in $\tilde{R}_p$ in the weighted case. Unlike the straight-line boundaries of standard
Voronoi diagrams, boundaries between regions in weighted Voronoi diagrams are hyperbolic arcs or straight lines \cite{aurenhammer_additive}.

A geometric object closely related to weighted Voronoi diagrams is the power diagram. Consider a Voronoi diagram where points in $S$ are replaced by
circles with possibly distinct radii. Instead of measuring the distance from a point in the plane to the center of the circles, we measure the distance
along lines tangent to the spheres to points in the plane. Using the Pythagorean Theorem, this is equivalent to measuring distances as $d^2(p,q) -
r^2(p)$ where $S$ contains a circle centered at $p$ of radius $r(p)$ \cite{aurenhammer_power}. This leads to a partition of the plane into regions defined by
\begin{equation*}
  \overline{R}_p = \{ q \in \R^2 | d^2(p,q) - r^2(p) \leq d^2(p',q) - r^2(p') \text{ for all } p' \in S \}.
\end{equation*}
This is known as the power diagram of $S$.

Power diagrams and additively weighted Voronoi diagrams have various connections to other geometric objects and each other.
Power diagrams are known to have connections to arrangements of hyperplanes in $\R^3$ ($\R^{d+1}$ in general) \cite{aurenhammer_survey}. The notions
of power diagrams and additively weighted Voronoi diagrams can be connected through embeddings of additively weighted Voronoi diagrams into
higher-dimensional power diagrams \cite{aurenhammer_additive}. This same connection holds for other generalized Voronoi diagrams.

\section{Algorithms}

The standard sweepline algorithm for computing Voronoi diagrams can be adapted to compute additively weighted Voronoi diagrams
\cite{fortune_sweepline}. Let $S$ be the set of points being considered with $w:S \to \R^+$ the associated weights. For this algorithm in the
unweighted case, a transformation is applied which maps $p \in S$ to the bottom of the region it is contained in, allowing a sweepline algorithm to
find regions as it reaches points in $S$. For the weighted Voronoi diagram, a modification of the transformation used takes weights into consideration
and allows the weighted Voronoi diagram to be computed. This leads to an algorithm requiring $O(n \log n)$ time and $O(n)$ space.

To compute a power diagram, relations between power diagrams in $\R^2$ and plane arrangemnts in $\R^3$ are used heavily \cite{aurenhammer_power}.
The following is my understanding of the diagram at this point, but it most likely has errors and is definitely incomplete: the algorithm uses a
transformation to map the circles in $\R^2$ to planes in $\R^3$. The planes generated by all of the given circles define the
convex hull of certain points in $\R^3$. By projecting the edges of this convex polyhedron back into $\R^2$, the power diagram is obtained. This algorithm runs
in $O(n \log n)$ time (in 2D).

\section{Applications of Power Diagrams}

Power diagrams have found applications in many diverse fields. Some of these uses are straightforward applications to computational geometry problems. Other
applications can be found in constrained clustering and numerical methods for fluid simulations.

Within computational geometry, power diagrams give useful algorithms for problems involving discs. One such problem is that of determining whether a point
lies within the union of a set of discs. Power diagrams also provide a way of determining the boundary of the union of a set of $n$ discs, a problem
that arises in computer graphics \cite{imai_power}. Another application of power diagrams is determining whether any discs within a set of $n$ discs overlap
\cite{aurenhammer_discs}. Given the $O(n \log n)$ algorithm for computing power diagrams, these problems can all be solved in $O(n \log n)$ time.

Power diagrams also find use in areas outside of algorithms. They can be used in clustering problems in which there are constraints on the way points
can be clustered \cite{brieden_clustering}. One example of this type of problem would be when a small number of farmers own many small plots of land
in the same area. To reduce operating costs, these farmers may agree to reallocate their lots in such a way that their lots are closer together. In
this reallocation process, the farmers are constrained because they need to group plots in such a way that each farmer gets roughly the same acreage
as before the reallocation \cite{brieden_farmland}.

%Power diagrams can also be used in manifold reconstruction. This problem takes a sampling of points from a smooth manifold in a higher-dimensional
%ambient space. The goal is to give a piecewise linear approximation of the surface of the original manifold. \textbf{I'M NOT SURE HOW POWER DIAGRAMS
%ARE BEING USED HERE}.

Power diagrams have also found use in some numerical methods for fluid equations. In simulations of fluids containing multiple substances using finite
volume methods, a single cell may end up containing multiple substances. After materials are located within a cell, power diagrams can be used to give an approximation to the
interface between the substances. This interface approximation allows the materials to be advected properly, leading to more accurate simulations
\cite{fluids}.

\section{$\alpha$-Shapes}

While working on the above topics, I will try to find some way to discuss $\alpha$-shapes in a way that results in a cohesive paper. $\alpha$-shapes give
a way of describing the ``shape'' of a set of points. One way of considering the shape of a set of points is using the convex hull. When thinking of
convex hulls, an edge in the convex hull must satisfy a sidedness condition involving a straight line for each point in the set. $\alpha$-shapes satisfy a similar sidedness condition
where $p$ and $q$ are connected by an edge if there is a circle of radius $\frac{1}{|\alpha|}$ with $p$ and $q$ on the boundary such that all other
points lie on the same side of this circle (whether we are looking at the interior or exterior of the circle is determined by the sign of $\alpha$). By
varying $\alpha$, a range of shapes can be obtained which include the points with no edges, the convex hull, and hulls that no longer need to be
convex and may some negative curvature \cite{edelsbrunner_3d} \cite{edelsbrunner_alpha}.

$\alpha$-shapes have connections to convex hulls and Delaunay triangulations \cite{edelsbrunner_alpha}. While these topics are related to Voronoi
diagrams, I do not plan on explicitly talking about them in my paper. Unless my plans change slightly, $\alpha$-shapes will not play a prominent role
in the paper I write because of the somewhat loose connection to other topics being presented.

\section{Proposed Survey}

The main emphasis for my paper will be studying power diagrams and additively weighted Voronoi diagrams. I plan to read and understand the background
information and the geometric tools for analyzing these topics and their algorithms through early March. By March 8th, I plan to have completed drafts
of sections in the paper detailing background information about power diagrams, additively weighted Voronoi diagrams, and their connections and the
algorithms used to compute each (see \cite{aurenhammer_power}, \cite{aurenhammer_survey}, \cite{aurenhammer_additive}, \cite{fortune_sweepline})
Following this I plan to study the applications of power diagrams to computational geometry and other fields (see \cite{aurenhammer_discs}, \cite{brieden_clustering},
\cite{brieden_farmland}, \cite{imai_power}, \cite{fluids}). I will also find and include reasonable applications of additively weighted Voronoi
diagrams, but I have not found any at this point. I plan to have a draft of the applications section added to the paper by March 29th. This
will give plenty of time before April 24th to edit and work on the presentation.

\printbibliography
\end{document}


