%        File: project.tex
%     Created: Sun Feb 12 08:00 PM 2017 C
% Last Change: Sun Feb 12 08:00 PM 2017 C
%

\documentclass[a4paper]{article}

\title{Power Diagrams and Additively Weighted Voronoi Diagrams Proposal}
\date{2/20/17}
\author{Trevor Steil}

\usepackage{amsmath}
\usepackage{amsthm}
\usepackage{amssymb}
\usepackage[backend=biber]{biblatex}
\usepackage{esint}
\usepackage{enumitem}
\usepackage{algorithm}
\usepackage{algorithmicx}
\usepackage{algpseudocode}
\usepackage{bbm}
\usepackage{xcolor}

\newtheorem{theorem}{Theorem}[section]
\newtheorem{corollary}{Corollary}[section]
\newtheorem{proposition}{Proposition}[section]
\newtheorem{lemma}{Lemma}[section]
\newtheorem*{claim}{Claim}
\newtheorem*{problem}{Problem}
%\newtheorem*{lemma}{Lemma}
\newtheorem{definition}{Definition}[section]

\newcommand{\R}{\mathbb{R}}
\newcommand{\N}{\mathbb{N}}
\newcommand{\C}{\mathbb{C}}
\newcommand{\Z}{\mathbb{Z}}
\newcommand{\Q}{\mathbb{Q}}
\newcommand{\E}{\mathbb{E}}
\newcommand{\supp}[1]{\mathop{\mathrm{supp}}\left(#1\right)}
\newcommand{\lip}[1]{\mathop{\mathrm{Lip}}\left(#1\right)}
\newcommand{\curl}{\mathrm{curl}}
\newcommand{\la}{\left \langle}
\newcommand{\ra}{\right \rangle}
\renewcommand{\vec}[1]{\mathbf{#1}}
\renewcommand{\div}{\mathrm{div}}

\newenvironment{solution}[1][]{\emph{Solution #1}}

\algnewcommand{\Or}{\textbf{ or }}
\algnewcommand{\And}{\textbf{ or }}

\addbibresource{project.bib}

\begin{document}
\maketitle

\section{Introduction}

The closest-neighbors Voronoi diagram for a set of points $S \subset \R^2$ partitions the plane into regions where each region has an associated $p \in S$ and
is defined by
\begin{equation*}
  R_p = \{ q \in \R^2 | d(p,q) \leq d(p',q) \text{ for all } p' \in S \} .
\end{equation*}

There are several ways to generalize this concept to give other interesting partitions of the plane. Partitioning into regions defined by sharing the
same $k$ closest points in $S$ gives rise to an order-$k$ Voronoi diagram \cite{aurenhammer_survey}. Another method for producing variations on
Voronoi diagrams is to consider using weighted distance functions. Let $w: S \to \R^+$ be a function that defines the weight of points of $S$. The
weighted Voronoi diagram partitions the plane into regions defined by
\begin{equation*}
  \tilde{R}_p = \{ q \in \R^2 | d(p,q) - w(p) \leq d(p',q) - w(p') \text{ for all } p' \in S \}.
\end{equation*}
It is worth noting that $p$ is not necessarily contained in $\tilde{R}_p$ in the weighted case. Unlike the straight-line boundaries of standard
Voronoi diagrams, boundaries between regions in weighted Voronoi diagrams are hyperbolic in shape \cite{aurenhammer_additive}.

A geometric object closely related to weighted Voronoi diagrams is the power diagram. Consider a Voronoi diagram where points in $S$ are replaced by
circles with possibly distinct radii. Instead of measuring the distance from a point in the plane to the center of the spheres, measure the distance
along lines tangent to the spheres to points in the plane. Using the Pythagorean Theorem, this is equivalent to measuring distances as $d^2(p,q) -
r^2(p)$ where $S$ contains a circle centered at $p$ of radius $r$. This leads to a partition of the plane into regions defined by
\begin{equation*}
  \overline{R}_p = \{ q \in \R^2 | d^2(p,q) - r^2(p) \leq d^2(p',q) - r^2(p') \}.
\end{equation*}
This is known as the power diagram of $S$.

\printbibliography
\end{document}


