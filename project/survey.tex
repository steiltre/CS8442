%        File: survey.tex
%     Created: Thu Mar 16 10:00 PM 2017 C
% Last Change: Thu Mar 16 10:00 PM 2017 C
%

\documentclass[a4paper, 11pt]{article}

\title{Power Diagrams and Additively Weighted Voronoi Diagrams }
\date{4/24/17}
\author{Trevor Steil}

\usepackage{amsmath}
\usepackage{amsthm}
\usepackage{amssymb}
\usepackage[backend=biber,style=numeric,citestyle=numeric,style=numeric]{biblatex}
\usepackage[margin=1in]{geometry}
\usepackage{esint}
\usepackage{enumitem}
\usepackage{algorithm}
\usepackage{algorithmicx}
\usepackage{algpseudocode}
\usepackage{bbm}
\usepackage{xcolor}

\newtheorem{theorem}{Theorem}[section]
\newtheorem{corollary}{Corollary}[section]
\newtheorem{proposition}{Proposition}[section]
\newtheorem{lemma}{Lemma}[section]
\newtheorem*{claim}{Claim}
%\newtheorem*{problem}{Problem}
%\newtheorem*{lemma}{Lemma}
\newtheorem{definition}{Definition}[section]

\newcommand{\R}{\mathbb{R}}
\newcommand{\N}{\mathbb{N}}
\newcommand{\C}{\mathbb{C}}
\newcommand{\Z}{\mathbb{Z}}
\newcommand{\Q}{\mathbb{Q}}
\newcommand{\E}{\mathbb{E}}
\newcommand{\supp}[1]{\mathop{\mathrm{supp}}\left(#1\right)}
\newcommand{\lip}[1]{\mathop{\mathrm{Lip}}\left(#1\right)}
\newcommand{\curl}{\mathrm{curl}}
\newcommand{\la}{\left \langle}
\newcommand{\ra}{\right \rangle}
\renewcommand{\vec}[1]{\mathbf{#1}}
\renewcommand{\div}{\mathrm{div}}

\newenvironment{problem}{\textbf{Problem.}}

\newenvironment{solution}[1][]{\emph{Solution #1}}

\algnewcommand{\Or}{\textbf{ or }}
\algnewcommand{\And}{\textbf{ or }}

\addbibresource{project.bib}

\begin{document}
\maketitle

\section{Introduction}
Voronoi diagrams provide a partitioning of space into regions where each region is defined by a \textit{site}. For a given site, its corresponding
region is defined as containing all points closer to that site than any other site. Voronoi diagrams have been studied as mathematical objects and as
computational tools providing solutions to many applied problems. Many generalizations and variations of Voronoi diagrams have also been studied in
similar ways.

\subsection{Definitions}

The closest-neighbors Voronoi diagram for a set of $n$ sites $S \subset \R^2$ partitions the plane into regions, or Voronoi cells, where each cell is associated to a
site $p \in S$ and is defined by
\begin{equation*}
  Vor(p) = \{ q \in \R^2 \ | \ d(p,q) < d(p',q) \text{ for all } p' \neq p \in S \} .
\end{equation*}
where $d(p,q)$ is the Euclidean distance between $p$ and $q$.

There are several ways to generalize this concept to give other interesting partitions of the plane. Partitioning into regions defined by sharing the
same $k$ closest sites in $S$ gives rise to an order-$k$ Voronoi diagram \cite{aurenhammer_survey}. When $k=n-1$, this gives the furthest-neighbor
Voronoi diagram.

Many other variations of Voronoi diagrams can be constructed through the use of weighted distance functions. Let $w: S \to \R^+$ be a function that defines
the weight of sites in $S$. The multiplicatively weighted Voronoi diagram partitions the plane into cells defined by
\begin{equation*}
  Vor_{mul}(p) = \left\{ q \in \R^2 \ \big| \ \frac{d(p,q)}{w(p)} < \frac{d(p',q)}{w(p')} \text{ for all } p' \neq p \in S \right\}.
\end{equation*}
These diagrams tend to be more difficult to study and are qualitatively very different from standard Voronoi diagrams. For example,
the boundaries between regions in multiplicatively weighted Voronoi diagrams form circular arcs \cite{ash-bolker}, and multiplicatively weighted Voronoi
cells can be disconnected and may partition the plane into $\Theta(n^2)$ connected components \cite{aurenhammer_survey}.

The additively weighted Voronoi diagram partitions the plane into cells defined by
\begin{equation*}
  Vor_{add}(p) = \left\{ q \in \R^2 \ | \ d(p,q) + w(p) < d(p',q) + w(p') \text{ for all } p' \neq p \in S \right\}.
\end{equation*}

A geometric object defined similarly to weighted Voronoi diagrams is the power diagram, which partitions the plane into power cells defined by
\begin{equation*}
  Vor_{pow}(p) = \{ q \in \R^2 \ | \ d^2(p,q) - w(p) \leq d^2(p',q) - w(p') \text{ for all } p' \neq p \in S \}.
\end{equation*}
For consistency, we will use $Vor_{pow}$ to denote power cells even though they are not given the name of Voronoi diagrams.
The distance function used in power diagrams is called the \textit{power function} and is denoted
$pow(q,p)$ where $q \in \R^2$ and $p \in \R^2$ is the center of a circle with radius $r(p)$.

The definitions of additively weighted Voronoi diagrams and power diagrams given above required $w(p)$ to be positive. This requirement is not
necessary because the definition of these Voronoi cells remains the same if any constant is added to $w(p)$ for all $p$. Requiring weights to be at
least nonnegative will make some of the following geometric relations easier to visualize.

In the remaining sections, additively weighted Voronoi diagrams and power diagrams will be the main consideration. We will use the term weighted
Voronoi diagram to refer to additively weighted Voronoi diagrams. Also, the definitions given above naturally extend to higher dimensions. We will
consider diagrams in $\R^2$, but many ideas and results will hold in higher dimensions as well. Some of these extensions and the differences they
present will be highlighted.

\subsection{Basic Geometric Results}

The definition of additively weighted Voronoi diagrams has a useful geometric interpretation. For a site $p \in S$, let $\mathcal{D}_p$ be the disc of radius $w(p)$ centered at $p$. Then we
can define
\begin{equation*}
  Vor_{add}(p) = \left\{ q \in \R^2 \ | \ \max_{\tilde{p} \in \mathcal{D}_p} d(\tilde{p}, q) < \max_{\tilde{p}' \in \mathcal{D}_{p'}} d(\tilde{p}', q) \text{
  for all } p' \neq p \in S \right\},
\end{equation*}
that is, we can measure distances as the maximum distance to any point in the disc centered at a given site \cite{rosenberger_additive}.

We can similarly interpret power diagrams in terms of circles. For $p \in S$, consider a circle $C(p)$ of radius $r(p) = \sqrt{w(p)}$ centered at $p$. Using the
Pythagorean theorem, we can see for any point $q \in \R^2$, $pow(q,p) = d^2(p,q) - r^2(p)$ gives the distance from $q$ to the circle centered at $p$
measured along a line tangent to $C(p)$. Throughout the paper, we may speak of weighted Voronoi diagrams and power diagrams by thinking of $S$ as
containing a set of points in $\R^2$ with associated weights, or we may refer to $S$ as a set of circles with radii given by the appropriate weights.

It is worth noting that $p$ is not necessarily contained in $Vor_{add}(p)$ or $Vor_{pow}(p)$. This can be seen from the definition if $w(p)$ is taken
to be large relative to the distance between $p$ and another site $p'$. This can also be interpreted as the associated circles being very large.
Despite this difference, the Voronoi cells of weighted Voronoi diagrams and power diagrams are still connected, and there are $O(n)$ vertices, edges,
and faces in the diagrams \cite{aurenhammer_power}. \textbf{CHECK IF THIS IS TRUE FOR WEIGHTED VORONOI}

Unlike the straight-line boundaries of standard
Voronoi diagrams, boundaries between regions in additively weighted Voronoi diagrams are hyperbolic arcs or straight line segments
\cite{aurenhammer_additive}. The boundaries between power cells are straight lines and line segments like in the case of standard Voronoi diagrams.
\textbf{FIND REFERENCE}

Power diagrams and additively weighted Voronoi diagrams have various connections to other geometric objects and each other.
Power diagrams are known to have connections to arrangements of hyperplanes in $\R^3$ ($\R^{d+1}$ in general) \cite{aurenhammer_survey}. The notions
of power diagrams and additively weighted Voronoi diagrams can be connected through embeddings of additively weighted Voronoi diagrams into
higher-dimensional power diagrams \cite{aurenhammer_additive}. This same connection holds for other generalized Voronoi diagrams.

\section{Relationships to Other Geometric Structures}
\label{sec:geom_rel}
The various Voronoi diagrams defined above have many connections to other geometric structures. There is the obvious connection to Delaunay
triangulations, the dual of Voronoi diagrams, which we will not discuss. Frequently, Voronoi diagrams in the plane can be related to 3-dimensional
objects.

\subsection{Cone Arrangements}
\label{cone}
For motivation, we will first consider a standard Voronoi diagram. Take a site $p \in S$. Define
\[ C_p = \{ (x, y, d(p, (x,y)) \}. \]
This defines a cone with vertex at $p$ in the $xy$-plane. We can see that $reg(p)$ is the region in $\R^2$ corresponding to the height of $C_p$ being
less than the height of all other cones. If we look at the intersection of two cones for $p \neq p'$, we have
\[ C_p \cap C_{p'} = \{ (x,y,z) | z = d(p, (x,y)) = d(p', (x,y)) \}. \]
Projecting $C_p \cap C_{p'}$ into $\R^2$ gives the points in the plane that are equidistant to $p$ and $p'$. This projection therefore contains
$reg(p) \cap reg(p')$ in the Voronoi diagram.

A segment of this projection is not in $reg(p) \cap reg(p')$ if and only if the projected segment is contained in the interior of $reg(p'')$ for $p''
\neq p, p'$. This corresponds to $C_{p''}$ having a lower height at $C_p \cap C_{p'}$.

We can apply the same reasoning to weighted Voronoi diagrams, where we define
\[ C_p = \{ (x, y, d(p, (x,y)) + w(p) \}. \]
This now gives cones that are shifted vertically by $w(p)$ units. Now $reg_{add}(p)$ is given by the region where $C_p$ is lower than all other cones.
We can also use this to find boundaries between regions in order-$k$ weighted Voronoi diagrams by not looking at intersections of cones where no other
cone lies below the intersection, but by looking for intersections where $k-1$ other cones lie below the intersection \cite{rosenberger_additive}.

The visualization as cones allows us to easily see that it is not necessarily the case that $p \in reg_{add}(p)$. If we shift a cone up far enough, it
will lie completely above another cone, giving $p \notin reg_{add}(p)$. More specifically, if $w(p) > d(p,p')$ for some $p' \in S$, then $p \notin
reg_{add}(p)$.

\subsection{Weighted Voronoi Diagrams and Power Diagrams}

Weighted Voronoi diagrams can be relateed to power diagrams through an embedding of weighted Voronoi diagrams in $\R^2$ into power diagrams in $\R^3$.
This embedding process applies more generally and requires a few definitions.

To begin with, we will state the result for general distance functions. Take $S \subset \R^2$ to be our sites. Any function $f: \R^2 \times S \to \R$
can be considered as a distance function that allows us to measure the distance between points in the plane and any of the sites. Using this distance
function, we can construct a Voronoi diagram, $V(S,f)$.

Let $F: \R \to \R$ be a strictly increasing function. For $p,q \in S \subset \R^2$, we define
\[ cone_F(p) = \{ (x,y,z) | (x,y) \in \R^2, z \ F( f( (x,y), p) ) \}. \]

We call a diagram \textit{affinely transformable} if there is a function $F$ as above such that $cone_F(p) \cap cone_F(q)$ is contained in a plane in $\R^3$ for any distinct
$p,q \in S$.

Let $V$ be a Voronoi diagram in $\R^3$. We say the 2-dimensional Voronoi diagram $V(S,f)$ can be \textit{embedded} in $V$ if for some $F$ as above,
there is a region $C$ of $V$ for each $p \in S$ such that $proj( C \cap cone_F(p)) = reg(p)$, where we have used $reg(p)$ to mean the region defined
by the distance function $f$, not necessarily a region defined for a standard Voronoi diagram.

We have the following relation between affinely transformable diagrams and power diagrams:

\begin{theorem}
  \label{thm:affine}
  A Voronoi diagram $V(S,f)$ in $\R^2$ is affinely transformable if and only if it can be embedded into a power diagram in $\R^3$
  \cite{aurenhammer_additive}.
\end{theorem}

As an example, all weighted Voronoi diagrams in $\R^2$ can be embedded in power diagrams in $\R^3$. We take $f(x,p) = d(x,p) + w(p)$ to be the
distance function defining weighted Voronoi diagrams, and $F(x) = x$. In this case, $cone_F(p)$ is the cone described in Section \ref{cone} for $p \in
S$. As already noted, the separators between regions in weighted Voronoi diagrams are hyperbolic arcs. The intersections $cone_F(p) \cap cone_F(q)$
are these arcs (no longer being cut off when another cone is below the intersection \textbf{GET BETTER WAY OF DESCRIBING THIS}) projected back onto
the cones, which will each be contained in a single plane in $\R^3$.

By definition, we have that weighted Voronoi diagrams are affinely transformable by the above reasoning. By Theorem \ref{thm:affine}, weighted Voronoi
diagrams can be embedded in power diagrams in $\R^3$. In fact, the corresponding power diagram is given by sites $T = \{ (p, w(p)) | p \in S \}$ and
weights $\omega(p) = 2 w(p)^2$, where $S \subset \R^2$ and $w(p)$ are the sites and weights for the weighted Voronoi diagram. \textbf{CHECK IF THIS
WEIGHT IS CORRECT}

\subsection{Convex Hulls}

There is an exact correspondence between power diagrams in $\R^2$ and convex hulls in $\R^3$. To show this, we will modify the argument in
\cite{comp_geom} to hold for power diagrams rather than standard Voronoi diagrams by using tools from \cite{aurenhammer_power}. We wil first show the correspondence between power diagrams and
upper envelopes and then use a correspondence between upper envelopes and lower convex hulls. As we will see later, this correspondence can be useful
for algorithmically finding power diagrams \cite{aurenhammer_power}.

First, we will need a few definitions. Let $U$ be the unit paraboloid in $\R^3$ defined as $f(x) = x \cdot x$ for $x \in \R^2$.

Let $H$ be a finite set of nonvertical planes in $\R^3$. Considering the height of a plane as a function of coordinates in the plane, we denote points
on the plane $h \in H$ as $(x,y,h(x,y))$. We define the \textit{upper envelope} of $H$ to be the object obtained by taking the plane with highest
$z$-coordinate above every point in the plane $z = 0$, that is,
\[ \mathcal{UE}(H) = \{ (x,y,z) | x,y \in \R, \ z = \max_{h \in H} h(x,y) \} .\]

Consider a set of sites $P \subset \R^2$ with an associated radii $\{r(p) | p \in P\}$. For $p \in P$, we define $\Pi(p)$ to be the plane with height above a point
$x \in \R^2$ given by
\[ \Pi(p) : z = 2 p \cdot x - p^2 + r^2(p) ,\]
where $p^2$ is used to denote $p \cdot p$. It can be shown that $\Pi$ is a bijection between spheres in $\R^2$ and nonvertical planes in $\R^3$ which
intersect $U$. Intuitively, if we consider a site with $r(p) = 0$, then $\Pi(p)$ is tangent to $U$, fixing the normal vector to $\Pi(p)$. For $f(x) =
x \cdot x$, $\nabla f(x) = 2x$, so a plane with any nonhorizontal normal vector can be found. Adding arbitrary constants shifts the obtained planes, to allow any
nonvertical plane. In the definition of $\Pi(p)$, $r^2(p)$ is being added, which must be a positive number. This limits the only translations of
tangent planes of $U$ to be such that an intersection with $U$ is maintained.

Now, we can begin to prove the correspondence between power diagrams and upper envelopes. First, we show the usefulness of the unit paraboloid $U$ in
relation to the power function.

\begin{lemma}
  \label{lem:proj}
  Let $q \in \R^2$ and $p \in P$. Let $q'$ be the vertical projection of $q$ onto $U$ and $q''$ be the vertical projection of $q$ onto $h(p)$. Then
  $pow(q, p) = d(q, q') - d(q, q'')$.
\end{lemma}
\begin{proof}
  The projections to get $q'$ and $q''$ are both vertical projections, so we only need to find the heights of $q'$ and $q''$. By
  definition of $U$, the height of $q'$ is
  \[ q_z' = q^2.\]
  Also, by definition of $\Pi(p)$, the height of $q''$ is
  \[ q_z'' = 2p \cdot q - p^2 + r^2(p) . \]
  Therefore,
  \begin{align*}
    d(q,q') - d(q,q'') &= q'_z - q''_z \\
    &= q^2 - 2 p \cdot q + p^2 - r^2(p) \\
    &= (p-q)^2 - r^2(p) \\
    &= pow(q,p)
  \end{align*}
\end{proof}

Now we are able to prove the correspondence.

\begin{theorem}
  Let $H = \{ \Pi(p) | p \in S \}$. Then the projection of $\mathcal{UE}(H)$ on the plane $z=0$ is the power diagram of $S$.
  \label{thm:ue_to_pow}
\end{theorem}
\begin{proof}
  We must show $reg_{pow}(p)$ is exactly the projection of the facet $\mathcal{UE}(H)$ that is part of $\Pi(p)$.

  Take $q \in reg_{pow}(p)$. Then by assumption, $pow(q,p) \leq pow(q,\tilde{p})$ for all $\tilde{p} \in S$. To use Lemma \ref{lem:proj}, we let $q_p''$ be the
  vertical projection of $q$ onto $\Pi(p)$, and $q_{\tilde{p}}''$ be the vertical projection of $q$ onto $\Pi(\tilde{p})$. We notice the vertical projection
  of $q$ onto $U$ is independent of $p$ and $\tilde{p}$, so we leave this as $q'$. By Lemma \ref{lem:proj},
  \begin{align*}
    d(q,q') - d(q,q_p'') &= pow(q,p) \\
    &\leq pow(q,\tilde{p}) \\
    &= d(q,q') - d(q,q_{\tilde{p}}'')
  \end{align*}
  Therefore, $d(q,q_p'') \geq d(q, q_{\tilde{p}}'')$. Because the projections are vertical, $d(q,q_p'')$ and $d(q, q_{\tilde{p}}'')$ are the heights of
  $\Pi(p)$ and $\Pi(\tilde{p})$ above the $q$. Thus, the height $\Pi(p)$ is greater than or equal to the height of $\Pi(\tilde{p})$ above $q$ for all
  $\tilde{p} \in S$. This means $\Pi(p)$ is in $\mathcal{UE}(H)$ above $q$, as we needed to show.
\end{proof}

By using the fact that $\Pi$ is a bijection between spheres in $\R^2$ and upper envelopes of collections of nonvertical planes that intersect $U$. To
see this, take each such plane. Using $\Pi^{-1}$, a sphere can be obtained in $\R^2$. By applying Lemma \ref{lem:proj} similar to in the proof of
Theorem \ref{thm:ue_to_pow}, facets of the upper envelope correspond to regions in the power diagram. This gives the following result:

\begin{corollary}
  There is a one-to-one correspondence between power diagrams in $\R^2$ and upper envelopes of arrangements of planes in $\R^3$ that intersect $U$.
  \label{cor:ue_pow}
\end{corollary}

This completes the correspondence between upper envelopes in $\R^3$ and power diagrams in $\R^2$. Now we must show the duality of lower convex hulls
and upper envelopes in $\R^3$. Let $h: z = ax + by + c$ be a plane in $\R^3$. As in \cite{aurenhammer_power}, we define
\begin{equation}
  \Delta(h) = \left( a, b, -c \right).
  \label{eq:duality1}
\end{equation}
This defines a map from nonvertical planes in $\R^3$ to points in $\R^3$. For a point $p \in \R^3$, we define
\begin{equation}
  \Delta(p) = \bigcup_{h \supseteq p} \Delta(h),
  \label{eq:duality2}
\end{equation}
where $h$ is ranging over all nonvertical planes containing $p$. This defines a map from points in $\R^3$ to planes in $\R^3$. For a point $p = (p_1, p_2, p_3)$,
by writing out the general form of a plane through $p$, this is equivalent to
\begin{equation}
  \Delta(p) := (z = p_1 x + p_2 y - p_3).
  \label{eq:duality3}
\end{equation}

We are able to use the same definition as above to map lines to lines, that is, for a line $l$ in $\R^3$, we define
\begin{equation}
  \Delta(l) = \bigcup_{h \supseteq l} \Delta(l).
  \label{eq:duality4}
\end{equation}

From the forms of \eqref{eq:duality1} and \eqref{eq:duality3}, it is clear that $\Delta$ is an involution. These operations can be thought of as
taking the plane $h: z = ax + bx - c$ and switching from $(a,b,c)$ as being coefficients and $(x,y,z)$ as being variables to $(x,y,z)$ being
coefficients and $(a,b,c)$ as being variables. From here, we can see these transformations are incidence preserving and order preserving, that is, $p \in h$ if and only if
$\Delta(h) \in \Delta(p)$ and $p$ is above $h$ if and only if $\Delta(p)$ is above $\Delta(h)$. For these reasons, $\Delta$ gives a useful dual
correspndence between points and planes in $\R^3$.

Consider a set of points $P \subseteq \R^3$. $p \in P$ is on the lower convex hull of $S$ if and only if there is a plane $h$ with $p \in h$ and $h$
is above $p'$ for all $p' \in S$. Using the order-preserving property of the duality transformation, this occurs if and only if there is an $h$ with $\Delta(h) \in
\Delta(p)$ and $\Delta(p')$ is above $\Delta(h)$ for all $p' \in S$. This last statement characterizes a plane being in the upper envelope of a set of
planes, giving the duality of lower convex hulls and upper envelopes in $\R^3$.

Combining the correspondence between power diagrams and upper envelopes given in Corollary \ref{cor:ue_pow} with this duality, we have the following:

\begin{theorem}
  Power diagrams in $\R^2$ exactly correspond to convex hulls which arise from upper envelopes of nonvertical planes that intersect $U$.
  \label{thm:ch_pow}
\end{theorem}

All of the results above were stated for power diagrams in $\R^2$ and upper envelopes and convex hulls in $\R^3$. It is easy to see these results can
be extended to power diagrams in $\R^d$ and upper envelopes and convex hulls in $\R^{d+1}$.

\section{Algorithms}

\subsection{Fortune's Sweepline Algorithm}

We begin with a description of Fortune's algorithm for computing Voronoi diagrams. Following this, we will give interpretations of this algorithm in
terms of cone arrangements. From here we will describe the changes necessary to adapt Fortune's algorithm to computing an additively weighted Voronoi
diagram.

The obvious obstacle with developing a sweepline algorithm for computing Voronoi diagrams is that sites will be reached by a sweepline in the middle
of its Voronoi cell. Therefore, the algorithm must have some way of identifying the creation of a new cell before it has reached a site. One method of
doing this is to perform a transformation that maps a site to the bottom of its Voronoi cell. A sweepline algorithm could then identify new cells as
they are formed and compute a transformed Voronoi diagram. After this transformed Voronoi diagram has been completed, the Voronoi diagram can be
obtained by inverting the previous transformation considered \cite{fortune_sweepline}.

\subsubsection{Algorithm Description}
Another way of interpreting Fortune's algorithm is using a sweepline and a beach line. We will give an overview of this associated algorithm. Many
details and proofs will not be covered, but they can be found in \cite{comp_geom}. The idea behind this approach is to use a
horizontal sweepline, $l$, that sweeps down the plane, while the beach line follows behind the sweepline and determines the boundary between the portion of the Voronoi diagram that cannot be
changed by anything past $l$ and the region that can still change.

To understand the structure of the beach line, we must understand what points in the plane will not be affected by any sites past $l$.
Consider a single site $p_i$ and point $q$ above $l$. If $d(p,q) < d(q,l)$, where $d(q,l)$ is understood to be the distance between $q$ and its
orthogonal projection onto $l$, then
no site not already reached by $l$ can be closer to $q$ than $p_i$. Similarly, if $d(p,q) > d(q,l)$, then a site just
below the sweepline could be closer to $q$ than $p_i$. The boundary between these two regions is the set of points that are equidistant from $p_i$ and
$l$. Therefore, this boundary is a parabola. Considering all sites at once, $q$ can only be changed by a site below $l$ if $d(p,q) > d(q,l)$ for all
currently known sites $p$. Therefore, the beach line is given by the bottom of the parabolas obtained for each individual site.

If we look at the intersection of two parabolic arcs on the beach line, these points will be equidistant from the sites associated to each parabolic
arc and $l$. This means as the sweep line moves, the beach line changes behind it and traces out the edges of the Voronoi diagram. To compute the
Voronoi diagram, we then need to understand how the beach line changes as new parabolic arcs form and old parabolic arcs are absorbed by the others.

Parabolic arcs can only be added when $l$ reaches a new site. At this point, a degenerate parabolic arc is added, which is just a vertical line
extending from the new site to the beach line. As $l$ continues downward, this parabola will ``open up'' and contribute to the beach line. Parabolic
arcs are removed from the beach line when they shrink down to a point. This can also be thought of as the two neighboring arcs expanding until the
center arc is squeezed to a single point, call it $q$. Each of the three arcs described correspond to unique sites $p_i, p_j$, and $p_k$.
\textbf{DESCRIBE THIS BETTER} At the point in time when the center parabolic arc has been squeezed to the single point $q$, $q$ is equidistant from
$p_i, p_j$, and $p_k$. Also, by the definition of the parabolic arcs, $q$ is equidistant from $l$, and is in fact tangent to $l$. Therefore, $q$ is a
vertex of the Voronoi diagram.

From the above considerations, the only times the structure of the beach line changes are when $l$ reaches a new site and when a circle lying on top
of $l$ has three sites contributing to the beach line on its boundary. Call these two events site events and circle events, respectively. At a site
event, a new parabolic arc is added. If this parabolic arc splits a single arc in two, the two intersection points trace out the new Voronoi edge in
both directions as $l$ continues downward. If the vertical line created when a new site is reached intersects two parabolic arcs of the beach line, we
are in a site event which is also a circle event and will treat it as a circle event. At a circle event, multiple edges intersect in a vertex, and a
new edge is traced out by the remaining parabolic arcs.

These events are stored in an event queue, $Q$. Events are sorted by $y$-value. To store the structure of the beach line, a balanced binary search
tree, $T$, is constructed with the parabolic arcs stored at the leaves. Each arc is also associated with the
site it corresponds to. The internal nodes store the intersection points of the arcs of the beach line. To insert a new parabolic arc that is created
when the sweepline crosses a site $p$, a binary search is performed on $T$ using $p_x$, the $x$-coordinate of $p$, to determine where to place the new
parabolic arc. The intersection points stored at the nodes of $T$ are not directly stored because they depend on the position of the sweepline.
Instead, they are calculated as the search is performed. Every leaf of $T$ stores a pointer to the circle event in $Q$ which will remove the
associated arc from the beach line. If no such event exists or has not been calculated yet, this pointer is set to \texttt{nil}.

The Voronoi diagram is stored in a DCEL structure as it is built. Internal nodes in $T$ contain pointers to one of the half-edges in the DCEL it is
updating. After the Voronoi diagram has been constructed, a large bounding box is added to give a valid DCEL structure.

\subsubsection{Interpretation Using Cone Arrangements}

The beach line used in Fortune's algorithm can be seen as the boundary of regions where sites below the sweepline can affect the Voronoi diagram being
computed. It also has a nice interpretation in terms of the cone arrangements previously mentioned that makes it clear how to generalize Fortune's
algorithm for computing additively weighted Voronoi diagrams.

For Voronoi diagrams, the cone arrangements involve cones with their vertices at the sites in the $xy$-plane because a Voronoi diagram is an
additively weighted Voronoi diagram with all weights being zero. Consider the sweepline from Fortune's algorithm as a plane in $\R^3$ that intersects
the $xy$-plane in a horizontal line. If this plane is given by the equation $z = y + t$, the plane is parallel to the boundary of the cones.
Therefore, the intersection of the sweep plane with the $xy$-plane is the sweepline from
Fortune's algorithm, where $t$ is the parameter used to make the plane sweep over $\R^3$. Also, the intersection of this plane with the cones of the
cone arrangement are exactly the parabolic arcs which give the beach line from above when projected onto the $xy$-plane. For this reason, Fortune's algorithm can best be thought of as a sweep-plane
algorithm over the cone arrangement associated to sites. The Voronoi edges traced out by intersections of the parabolic arcs of the beach line are
these parabolas on the cones intersecting to trace out the Voronoi edges of the lifted diagram.

\subsubsection{Adaptation to Weighted Voronoi Diagrams}
From the cone arrangement interpretation, we can see that the appropriate algorithm is the same sweep plane algorithm where the cone at site $p$ is
now shifted up by the site's weight, $w(p)$ \cite{rosenberger_additive}. This can also be reinterpreted as a sweepline algorithm in the plane as
above.

Instead of sites located at points, we now consider site discs of radius $w(p_i)$ centered at site $p(i)$. Parabolas are added to the beach
line when the sweep line reaches the bottom of a site's disc. This corresponds to when the sweep plane reaches the vertex of the associated cone. The
parabola that is added has its focus at the site $p_i$, and its directrix begins touching the focus, just as in Fortune's algorithm, that is, the
directrix for a given parabola lags behind the sweepline by a distance of $w(p_i)$. It is worth noting that in this situation, a parabola may be added
when the associated site is already behind the beach line, creating a situation where a new Voronoi cell is not added.

As before, we do not have to explicitly track these parabolas. The structure of the beach line only changes at site events and circle events. Site
events are described above and potentially add new parabolas to the beach line. Circle events are defined similarly to the circle events for
unweighted Voronoi diagrams, but now a circumcircle must lie tangent to the sweepline and be tangent to the site discs associated to three adjacent
parabolas. In this case, the center of the circumcircle is a vertex of the weighted Voronoi diagram, and the middle parabola is removed.

The same priority queue, $Q$, balanced binary search tree, $T$, and DCEL structure can be used to store the events that change the beach line's
structure, the tree that stores the structure of the beach line, and the weighted Voronoi diagram being constructed. The DCEL structure returned will
not be exactly the weighted Voronoi diagram because the DCEL will only contain straight edges, whereas a weighted Voronoi diagram can have edges that
are hyperbolic arcs. The DCEL returned will only contain connectivity information for the diagram. \textbf{CAN WE STORE EXTRA VALUES TO GIVE THE
ACTUAL DIAGRAM??}

\subsubsection{Computational Complexity}

At each event, the DCEL structure being stored can be updated in $O(1)$ time. $T$ and $Q$ can both be updated in $\log(n)$ time. Each site event corresponds to a site in $S$, giving a $O(n \log n)$ term from site events. Circle
events are slightly more difficult to count, but there are $O(n)$ circle events \cite{comp_geom}. Combining the site and circle events gives a total
computational complexity of $O(n \log n)$.

\subsection{Computing Power Diagrams}

As in using Fortune's algorithm for computing additively weighted Voronoi diagrams, we will be making use of the geometric relations given in
Section \ref{sec:geom_rel} to compute power diagrams. In this case, we will be using these relations in a much more direct way. To compute the power
diagram of $S$, we can compute a lower convex hull by Theorem \ref{thm:ch_pow} and use the relations between lower convex hulls, upper envelopes and
power diagrams to extract a power diagram \cite{aurenhammer_power} . This gives the following algorithm

\begin{algorithm}
  \begin{algorithmic}[1]
    \Function{PowerDiagram}{S}
    \State Compute the hyperplanes $\Pi(S) = \{ \Pi(p) | p \in S \}$ and points $\Delta( \Pi(S) ) = \{ \Delta( \Pi(p) ) | p \in S \}$
    \State Compute the convex hull $CH( \Delta( \Pi(S) ) )$.
    \State Compute the upper envelope $UE( \Pi(S) )$ by applying $\Delta$ to all vertices, edges, and faces of $CH( \Delta( \Pi(S) ) )$.
    \State Project all edges of $UE( \Pi(S) )$ onto $\R^2$ to obtain $PD(S)$.
    \EndFunction
  \end{algorithmic}
\end{algorithm}

\subsubsection{Computational Complexity}

Most of the steps for computing the power diagram involve applying transformations to sets of size $O(n)$. The only step that takes more time is
computing the convex hull, which can be done in $O(n \log n)$ time \cite{comp_geom}. Therefore, the above gives an algorithm for computing power
diagrams in $O(n \log n)$ time.

\section{Applications of Weighted Voronoi Diagrams and Power Diagrams}

\subsection{Point in Union of Balls}

Let $B = \{b_1,\dots, b_n\}$ be a collection of open balls and $S = \{s_1, \dots, s_n\}$ be the circles bounding each of the balls in $B$. We would
like to be able to efficiently determine if a query point $q$ lies in $\cup(B) = \cup_{i=1}^n b_i$. This query can be answered by using the power diagram
built on the circles of $S$ \cite{aurenhammer_discs}. The main necessary result is the following.

\begin{lemma}
  \label{lem:pow_query}
  $x \in \cup B$ if and only if $x \in Vor_{pow}(s_i) \cap b_i$ for some $i$.
\end{lemma}
\begin{proof}
  Let $x \in Vor_{pow}(s_i) \cap b_i$. Then $x \in \cup B$ by assumption.

  Now assume $x \in \cup B$. Assume $x \in b_j$. The power diagram partitions $\R^2$, so $x \in Vor_{pow}(s_i)$ for some $i$. We must show $x \in
  b_i$. By definition of the power function, $x \in b_k \Leftrightarrow pow(x,s_k) \leq 0$ for any $k$. Because $x \in b_j$, $pow(x,s_j) \leq 0$. $x
  \in Vor_{pow}(s_i)$, so
  \begin{align*}
    pow(x,s_i) &\leq pow(x, s_j) \\
    &\leq 0
  \end{align*}
  Therefore, $x \in s_i$.
\end{proof}

With this result, answering the query problem can be reduced to determining whether $q \in Vor_{pow}(s_i) \cap b_i$ for all $i$. This is done by
solving a planar point location problem on the power diagram of $S$ with $q$ as the query point. This will return a single region with an associated
circle $s_i$. By Lemma \ref{lem:pow_query}, $x \in \cup B$ if and only if $x \in b_i$ for this particular value of $i$.

The time necessary to perform this query is dominated by the planar point location problem involved, which is $O(\log n)$. After this, a single
intersection test is performed on the necessary circle. This reduces the query time of the naive method of querying each ball in $B$, which has a
computational complexity of $O(n)$.

It is worth noting that this algorithm answers the decision problem. This exact method will not be able to return the entire set of $b_i$ that $q$
intersects. The planar point location only returns a single circle in $S$ to check against. This can be overcome with methods of colored intersection
searching \cite{ravi}. This is accomplished by considering each circle in $S$ to have a unique color. Each power diagram has size $O(n)$, and we have
an algorithm for solving the decision problem in $O(\log n)$ time. By Theorem 1.8 in \cite{ravi}, we can output the results of the associated reporting problem using $O(n
\log n)$ space in $O(\log n +k \log^2 n)$ time, where $k$ is the number of reported balls of $B$.

\subsection{Detecting Intersection of Spheres}

Use fact that boundary between power cells is through weighted bisector of line between centers of circles to say circles intersect if and only if
they intersect the edge between cells. There are $O(n)$ total edges, each of which needs to be checked against two circles. Check in $O(n)$ time.

\subsection{Clustering}

For a set $S$ of $n$ points, $k$-clustering seeks to partition $S$ into $k$ nonempty subsets, or clusters, $S_1, \dots, S_k$ which minimizes some objective
function. When certain objective functions are used, optimal clusterings can be expressed in terms of Voronoi diagrams. For a cluster $S_i$ and a
parameter $\alpha$, define
the variance of $S_i$ as
\begin{equation}
  Var^\alpha(S_i) = |S_i|^{\alpha-1} \sum_{p \in S_i} \| p - \overline{p}(S_i) \|^2
  \label{def:var}
\end{equation}
where $\| \cdot \|$ denotes the Euclidean norm and $\overline{p}(S_i)$ is the centroid of $S_i$. The minimization we will consider is
\begin{equation}
  \min \left\{ \sum_{i=1}^k Var^\alpha (S_i) | (S_1,\dots, S_k) \text{ is a $k$-clustering of } S \right\}
  \label{eq:cluster}
\end{equation}
for $\alpha = 1,2$.

Consider the case of $\alpha = 1$. The clustering problem is then seeking to minimize the sum of squared distances from points of $S$ to the
associated centroids. If we know the centroids associated to each cluster, this becomes a matter of associating each point of $S$ to the nearest
centroid, which is the same as drawing the standard Voronoi diagram for the centroids and determining the Voronoi cell each point of $S$ belongs to.
This results in the following

\begin{theorem}
  \label{thm:var_1}
  Suppose $(S_1^\ast, \dots, S_k^\ast)$ is an optimal $k$-clustering for minimizing $\sum_{i=1}^k Var^1(S_i)$. Let $p_j^\ast$ be the associated
  centroids for each cluster. Consider the Voronoi diagram obtained by using $\{ p_j^\ast \}$ as sites. Then $S_j = S \cap Vor(p_j^\ast)$, where
  $Vor(p_j^\ast)$ is the Voronoi cell associated to $p_j^\ast$ \cite{inaba_clustering} .
\end{theorem}

A similar theorem holds when minimizing $\sum_{i=1}^k Var^2(S_i)$, but now a Voronoi diagram that is weighted additively and multiplicatively must be
considered. For this result, we will consider sites $\{ p_j \}$ with multiplicative weights $\nu_j$ and additive
weights $\sigma_j$ for $j=1,\dots, k$. We define the additively and multiplicatively weighted Voronoi cell for a site $\overline{p}(S_j)$ as
\begin{equation}
  Vor_{wgt}( p_j ) = \{ p \ | \ \nu_j \| p - p_j \|^2 + \sigma_j \leq \nu_l \| p - p_l \|^2 + \sigma_l \text{
    for all $l \neq j$} \}
  \label{def:wgt_vor}
\end{equation}
To avoid confusion with the convention of referring to additively weighted Voronoi diagrams as weighted Voronoi diagrams throughout this paper, we
will use the slightly cumbersome name of additively and multiplicatively weight Voronoi diagrams to refer to the diagrams given by these cells. With
this definition, we have the following result.

\begin{theorem}
  \label{thm:var_2}
  Suppose $(S_1^\ast, \dots, S_k^\ast)$ is an optimal $k$-clustering for minimizing $\sum_{i=1}^k Var^2(S_i)$. Let $p_j^\ast$ be the associated
  centroids for each cluster. Consider the additively and multiplicatively weighted Voronoi diagram obtained by using $\{ p_j^\ast \}$ as sites with
  additive weights $\nu_j = |S_j^\ast|$ and multiplicative weights $\sigma_j = \sum_{p_i \in S_j^\ast} \| p_i - p_j^\ast \|^2$. Then $S_j^\ast =
  S \cap Vor_{wgt}(p_j^\ast)$ for every $j$ \cite{inaba_clustering}.
\end{theorem}

The proof of Theorem \ref{thm:var_2} makes use of the fact that $Var^2(S_j)$ is the all-pairs sum of squared errors, that is
\begin{equation}
  Var^2(S_j) = \sum_{p_i, p_l \in S_j, i<l} \|p_i - p_l\|^2.
  \label{eq:all_sums}
\end{equation}
Similar to the proof of this fact, it can be shown that for any point $p$,
\begin{equation}
  \sum_{p_i \in S_j} \| p_i - p \|^2 = |S_j| \|p - p_j^\ast\|^2 + \sum_{p_i \in S_j} \| p_i - p_j^\ast \|^2.
  \label{eq:wgt_vor}
\end{equation}
Combining these two facts, a contradiction can be reached by assuming a point $p_i \in S_j$ but $p_i \not\in Vor_{wgt}(p_j^\ast)$ by showing that
moving $p_i$ to $Vor_{wgt}(p_j^\ast)$ will decrease the weighted distance function used to create the additively and multiplicatively weighted Voronoi
diagram.

With Theorem \ref{thm:var_1} and Theorem \ref{thm:var_2}, a solution to these clustering problems can be found by finding all possible Voronoi
partitions of $S$ using $k$ weighted points and then choosing the partition that minimizes the objective function.

\printbibliography
\end{document}


