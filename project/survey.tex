%        File: survey.tex
%     Created: Thu Mar 16 10:00 PM 2017 C
% Last Change: Thu Mar 16 10:00 PM 2017 C
%

\documentclass[12pt]{article}

\title{Power Diagrams and Additively Weighted Voronoi Diagrams }
\date{4/24/17}
\author{Trevor Steil}

\usepackage{amsmath}
\usepackage{amsthm}
\usepackage{amssymb}
\usepackage[backend=biber,style=numeric,citestyle=numeric,style=numeric]{biblatex}
\usepackage{esint}
\usepackage{enumitem}
\usepackage{algorithm}
\usepackage{algorithmicx}
\usepackage{algpseudocode}
\usepackage{bbm}
\usepackage{xcolor}

\newtheorem{theorem}{Theorem}[section]
\newtheorem{corollary}{Corollary}[section]
\newtheorem{proposition}{Proposition}[section]
\newtheorem{lemma}{Lemma}[section]
\newtheorem*{claim}{Claim}
%\newtheorem*{problem}{Problem}
%\newtheorem*{lemma}{Lemma}
\newtheorem{definition}{Definition}[section]

\newcommand{\R}{\mathbb{R}}
\newcommand{\N}{\mathbb{N}}
\newcommand{\C}{\mathbb{C}}
\newcommand{\Z}{\mathbb{Z}}
\newcommand{\Q}{\mathbb{Q}}
\newcommand{\E}{\mathbb{E}}
\newcommand{\supp}[1]{\mathop{\mathrm{supp}}\left(#1\right)}
\newcommand{\lip}[1]{\mathop{\mathrm{Lip}}\left(#1\right)}
\newcommand{\curl}{\mathrm{curl}}
\newcommand{\la}{\left \langle}
\newcommand{\ra}{\right \rangle}
\renewcommand{\vec}[1]{\mathbf{#1}}
\renewcommand{\div}{\mathrm{div}}

\newenvironment{problem}{\textbf{Problem.}}

\newenvironment{solution}[1][]{\emph{Solution #1}}

\algnewcommand{\Or}{\textbf{ or }}
\algnewcommand{\And}{\textbf{ or }}

\addbibresource{project.bib}

\begin{document}
\maketitle

\section{Introduction}
Voronoi diagrams provide a partitioning of space into regions where each region is defined by a \textit{site}. For a given site, its corresponding
region is defined as containing all points closer to that site than any other site. Voronoi diagrams have been studied as mathematical objects and as
computational tools providing solutions to many applied problems. Many generalizations and variations of Voronoi diagrams have also been studied in
similar ways.

\subsection{Definitions}

The closest-neighbors Voronoi diagram for a set of $n$ sites $S \subset \R^2$ partitions the plane into regions where each region is associated to a
site $p \in S$ and is defined by
\begin{equation*}
  reg(p) = \{ q \in \R^2 | d(p,q) \leq d(p',q) \text{ for all } p' \in S \} .
\end{equation*}
where $d(p,q)$ is the Euclidean distance between $p$ and $q$.

There are several ways to generalize this concept to give other interesting partitions of the plane. Partitioning into regions defined by sharing the
same $k$ closest sites in $S$ gives rise to an order-$k$ Voronoi diagram \cite{aurenhammer_survey}. When $k=n-1$, this gives the furthest-neighbor
Voronoi diagram.

Many other variations of Voronoi diagrams can be constructed through the use of weighted distance functions. Let $w: S \to \R^+$ be a function that defines
the weight of sites in $S$. The multiplicatively weighted Voronoi diagram partitions the plane into regions defined by
\begin{equation*}
  reg_{mul}(p) = \left\{ q \in \R^2 | \frac{d(p,q)}{w(p)} \leq \frac{d(p',q)}{w(p')} \text{ for all } p' \in S \right\}.
\end{equation*}
The boundaries between regions in multiplicatively weighted Voronoi diagrams form circular arcs \cite{ash-bolker}.

The additively weighted Voronoi diagram partitions the plane into regions defined by
\begin{equation*}
  reg_{add}(p) = \left\{ q \in \R^2 | d(p,q) + w(p) \leq d(p',q) + w(p') \text{ for all } p' \in S \right\}.
\end{equation*}
This definition also has a geometric interpretation. For a site $p \in S$, let $\mathcal{D}_p$ be the disc of radius $w(p)$ centered at $p$. Then we
can define
\begin{equation*}
  reg_{add}(p) = \left\{ q \in \R^2 | \max_{\tilde{p} \in \mathcal{D}_p} d(\tilde{p}, q) \leq \max_{\tilde{p}' \in \mathcal{D}_{p'}} d(\tilde{p}', q) \text{
  for all } p' \in S \right\},
\end{equation*}
that is, we can measure distances as the maximum distance to any point in the disc centered at a given site \cite{rosenberger_additive}.

It is worth noting that $p$ is not necessarily contained in $\tilde{R}_p$ in the additively weighted case. Unlike the straight-line boundaries of standard
Voronoi diagrams, boundaries between regions in additively weighted Voronoi diagrams are hyperbolic arcs or straight line segments \cite{aurenhammer_additive}.

A geometric object defined similarly to weighted Voronoi diagrams is the power diagram. Consider a Voronoi diagram where sites in $S$ are replaced by
circles with possibly distinct radii. Instead of measuring the distance from a point in the plane to the center of the circles, we measure the distance
along lines tangent to the circles. Using the Pythagorean Theorem, this is equivalent to measuring distances as $d^2(p,q) -
r^2(p)$ where $S$ contains a circle centered at $p$ of radius $r(p)$ \cite{aurenhammer_power}. This leads to a partition of the plane into regions defined by
\begin{equation*}
  reg_{pow}(p) = \{ q \in \R^2 | d^2(p,q) - r^2(p) \leq d^2(p',q) - r^2(p') \text{ for all } p' \in S \}.
\end{equation*}
This is known as the power diagram of $S$. The distance function used in power diagrams is called the \textit{power function} and is denoted
$pow(q,p)$ where $q \in \R^2$ and $p \in \R^2$ is the center of a circle with radius $r(p)$.

Power diagrams and additively weighted Voronoi diagrams have various connections to other geometric objects and each other.
Power diagrams are known to have connections to arrangements of hyperplanes in $\R^3$ ($\R^{d+1}$ in general) \cite{aurenhammer_survey}. The notions
of power diagrams and additively weighted Voronoi diagrams can be connected through embeddings of additively weighted Voronoi diagrams into
higher-dimensional power diagrams \cite{aurenhammer_additive}. This same connection holds for other generalized Voronoi diagrams.

In the remaining sections, additively weighted Voronoi diagrams and power diagrams will be the main consideration. We will use the term weighted
Voronoi diagram to refer to additively weighted Voronoi diagrams. Also, the definitions given above naturally extend to higher dimensions. We will
consider diagrams in $\R^2$, but many ideas and results will hold in higher dimensions as well.

\section{Relationships to Other Geometric Structures}
The various Voronoi diagrams defined above have many connections to other geometric structures. There is the obvious connection to Delaunay
triangulations, the dual of Voronoi diagrams, which we will not discuss. Frequently, Voronoi diagrams in the plane can be related to 3-dimensional
objects.

\subsection{Cone Arrangements}
\label{cone}
For motivation, we will first consider a standard Voronoi diagram. Take a site $p \in S$. Define
\[ C_p = \{ (x, y, d(p, (x,y)) \}. \]
This defines a cone with vertex at $p$ in the $xy$-plane. We can see that $reg(p)$ is the region in $\R^2$ corresponding to the height of $C_p$ being
less than the height of all other cones. If we look at the intersection of two cones for $p \neq p'$, we have
\[ C_p \cap C_{p'} = \{ (x,y,z) | z = d(p, (x,y)) = d(p', (x,y)) \}. \]
Projecting $C_p \cap C_{p'}$ into $\R^2$ gives the points in the plane that are equidistant to $p$ and $p'$. This projection therefore contains
$reg(p) \cap reg(p')$ in the Voronoi diagram.

A segment of this projection is not in $reg(p) \cap reg(p')$ if and only if the projected segment is contained in the interior of $reg(p'')$ for $p''
\neq p, p'$. This corresponds to $C_{p''}$ having a lower height at $C_p \cap C_{p'}$.

We can apply the same reasoning to weighted Voronoi diagrams, where we define
\[ C_p = \{ (x, y, d(p, (x,y)) + w(p) \}. \]
This now gives cones that are shifted vertically by $w(p)$ units. Now $reg_{add}(p)$ is given by the region where $C_p$ is lower than all other cones.
We can also use this to find boundaries between regions in order-$k$ weighted Voronoi diagrams by not looking at intersections of cones where no other
cone lies below the intersection, but by looking for intersections where $k-1$ other cones lie below the intersection \cite{rosenberger_additive}.

The visualization as cones allows us to easily see that it is not necessarily the case that $p \in reg_{add}(p)$. If we shift a cone up far enough, it
will lie completely above another cone, giving $p \notin reg_{add}(p)$. More specifically, if $w(p) > d(p,p')$ for some $p' \in S$, then $p \notin
reg_{add}(p)$.

\subsection{Weighted Voronoi Diagrams and Power Diagrams}

Weighted Voronoi diagrams can be relateed to power diagrams through an embedding of weighted Voronoi diagrams in $\R^2$ into power diagrams in $\R^3$.
This embedding process applies more generally and requires a few definitions.

To begin with, we will state the result for general distance functions. Take $S \subset \R^2$ to be our sites. Any function $f: \R^2 \times S \to \R$
can be considered as a distance function that allows us to measure the distance between points in the plane and any of the sites. Using this distance
function, we can construct a Voronoi diagram, $V(S,f)$.

Let $F: \R \to \R$ be a strictly increasing function. For $p,q \in S \subset \R^2$, we define
\[ cone_F(p) = \{ (x,y,z) | (x,y) \in \R^2, z \ F( f( (x,y), p) ) \}. \]

We call a diagram \textit{affinely transformable} if there is a function $F$ as above such that $cone_F(p) \cap cone_F(q)$ is contained in a plane in $\R^3$ for any distinct
$p,q \in S$.

Let $V$ be a Voronoi diagram in $\R^3$. We say the 2-dimensional Voronoi diagram $V(S,f)$ can be \textit{embedded} in $V$ if for some $F$ as above,
there is a region $C$ of $V$ for each $p \in S$ such that $proj( C \cap cone_F(p)) = reg(p)$, where we have used $reg(p)$ to mean the region defined
by the distance function $f$, not necessarily a region defined for a standard Voronoi diagram.

We have the following relation between affinely transformable diagrams and power diagrams:

\begin{theorem}
  \label{thm:affine}
  A Voronoi diagram $V(S,f)$ in $\R^2$ is affinely transformable if and only if it can be embedded into a power diagram in $\R^3$
  \cite{aurenhammer_additive}.
\end{theorem}

As an example, all weighted Voronoi diagrams in $\R^2$ can be embedded in power diagrams in $\R^3$. We take $f(x,p) = d(x,p) + w(p)$ to be the
distance function defining weighted Voronoi diagrams, and $F(x) = x$. In this case, $cone_F(p)$ is the cone described in Section \ref{cone} for $p \in
S$. As already noted, the separators between regions in weighted Voronoi diagrams are hyperbolic arcs. The intersections $cone_F(p) \cap cone_F(q)$
are these arcs (no longer being cut off when another cone is below the intersection \textbf{GET BETTER WAY OF DESCRIBING THIS}) projected back onto
the cones, which will each be contained in a single plane in $\R^3$.

By definition, we have that weighted Voronoi diagrams are affinely transformable by the above reasoning. By Theorem \ref{thm:affine}, weighted Voronoi
diagrams can be embedded in power diagrams in $\R^3$. In fact, the corresponding power diagram is given by sites $T = \{ (p, w(p)) | p \in S \}$ and
weights $\omega(p) = 2 w(p)^2$, where $S \subset \R^2$ and $w(p)$ are the sites and weights for the weighted Voronoi diagram. \textbf{CHECK IF THIS
WEIGHT IS CORRECT}

\subsection{Convex Hulls}

There is an exact correspondence between power diagrams in $\R^2$ and convex hulls in $\R^3$. To show this, we will modify the argument in
\cite{comp_geom} to hold for power diagrams rather than standard Voronoi diagrams by using tools from \cite{aurenhammer_power}. We wil first show the correspondence between power diagrams and
upper envelopes and then use a correspondence between upper envelopes and lower convex hulls. As we will see later, this correspondence can be useful
for algorithmically finding power diagrams \cite{aurenhammer_power}.

First, we will need a few definitions. Let $U$ be the unit paraboloid in $\R^3$ defined as $z = x^2 + y^2$.

Let $H$ be a finite set of nonvertical planes in $\R^3$. Considering the height of a plane as a function of coordinates in the plane, we denote points
on the plane $h \in H$ as $(x,y,h(x,y))$. We define the \textit{upper envelope} of $H$ to be the object obtained by taking the plane with highest
$z$-coordinate above every point in the plane, that is,
\[ \mathcal{UE}(H) = \{ (x,y,z) | x,y \in \R, \ z = \max_{h \in H} h(x,y) \} .\]

Consider a set of sites $P \subset \R^2$ with an associated radius $r(p)$ for $p \in P$. For $p \in P$, we define $h(p)$ to be the plane with height above a point
$q \in \R^2$ given by
\[ z = 2 p \cdot q - p^2 + r^2(p) ,\]
where $p^2$ is used to denote $p \cdot p$.

Now, we can begin to prove the correspondence between power diagrams and upper envelopes. First, we show the usefulness of the unit paraboloid $U$ in
relation to the power function.

\begin{lemma}
  \label{lem:proj}
  Let $q \in \R^2$ and $p \in P$. Let $q'$ be the vertical projection of $q$ onto $U$ and $q''$ be the vertical projection of $q$ onto $h(p)$. Then
  $pow(q, p) = d(q, q') - d(q, q'')$.
\end{lemma}
\begin{proof}
  See \cite{aurenhammer_power}
\end{proof}

Now we are able to prove the correspondence.

\begin{theorem}
  Let $H = \{ h(p) | p \in P \}$. Then the projection of $\mathcal{UE}(H)$ on the plane $z=0$ is the power diagram of $P$.
\end{theorem}
\begin{proof}
  We must show $reg_{pow}(p)$ is exactly the projection of the facet $\mathcal{UE}(H)$ that is part of $h(p)$.

  Take $q \in reg_{pow}(p)$. Then by assumption, $pow(q,p) \leq pow(q,\tilde{p})$ for all $\tilde{p} \in P$. To use Lemma \ref{lem:proj}, we let $q_p''$ be the
  vertical projection of $q$ onto $h(p)$, and $q_{\tilde{p}}''$ be the vertical projection of $q$ onto $h(\tilde{p})$. We notice the vertical projection
  of $q$ onto $U$ is independent of $p$ and $\tilde{p}$, so we leave this as $q'$. By Lemma \ref{lem:proj},
  \begin{align*}
    d(q,q') - d(q,q_p'') &= pow(q,p) \\
    &\leq pow(q,\tilde{p}) \\
    &= d(q,q') - d(q,q_{\tilde{p}}'')
  \end{align*}
  Therefore, $d(q,q_p'') \geq d(q, q_{\tilde{p}}'')$. Because the projections are vertical, $d(q,q_p'')$ and $d(q, q_{\tilde{p}}'')$ are the heights of
  $h(p)$ and $h(\tilde{p})$ above the $q$. Thus, the height $h(p)$ is greater than or equal to the height of $h(\tilde{p})$ above $q$ for all
  $\tilde{p} \in P$. This means $h(p)$ is in $\mathcal{UE}(H)$ above $q$, as we needed to show.
\end{proof}

This completes the correspondence between upper envelopes in $\R^3$ and power diagrams in $\R^2$. Using the duality of upper envelopes and lower convex hulls (see
\cite{comp_geom}), we get the final correspondence between convex hulls in $\R^3$ and power diagrams in $\R^2$.

\section{Algorithms}

\printbibliography
\end{document}


