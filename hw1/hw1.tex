%        File: hw1.tex
%     Created: Mon Jan 23 07:00 PM 2017 C
% Last Change: Mon Jan 23 07:00 PM 2017 C
%

\documentclass[a4paper]{article}

\title{CSci 8442 Homework 1 }
\date{2/13/17}
\author{Trevor Steil}

\usepackage{amsmath}
\usepackage{amsthm}
\usepackage{amssymb}
\usepackage{esint}
\usepackage{enumitem}
\usepackage{algorithm}
\usepackage{algorithmicx}
\usepackage{algpseudocode}
\usepackage{bbm}
\usepackage{xcolor}

\newtheorem{theorem}{Theorem}[section]
\newtheorem{corollary}{Corollary}[section]
\newtheorem{proposition}{Proposition}[section]
\newtheorem{lemma}{Lemma}[section]
\newtheorem*{claim}{Claim}
%\newtheorem*{problem}{Problem}
%\newtheorem*{lemma}{Lemma}
\newtheorem{definition}{Definition}[section]

\newcommand{\R}{\mathbb{R}}
\newcommand{\N}{\mathbb{N}}
\newcommand{\C}{\mathbb{C}}
\newcommand{\Z}{\mathbb{Z}}
\newcommand{\Q}{\mathbb{Q}}
\newcommand{\E}{\mathbb{E}}
\newcommand{\supp}[1]{\mathop{\mathrm{supp}}\left(#1\right)}
\newcommand{\lip}[1]{\mathop{\mathrm{Lip}}\left(#1\right)}
\newcommand{\curl}{\mathrm{curl}}
\newcommand{\la}{\left \langle}
\newcommand{\ra}{\right \rangle}
\renewcommand{\vec}[1]{\mathbf{#1}}
\renewcommand{\div}{\mathrm{div}}

\newenvironment{problem}{\textbf{Problem.}}

\newenvironment{solution}[1][]{\emph{Solution #1}}

\algnewcommand{\Or}{\textbf{ or }}
\algnewcommand{\And}{\textbf{ or }}

\begin{document}
\maketitle

\begin{enumerate}
  \item
    \begin{problem}
      Let $S = \{ p_1, \dots, p_n \}$ be a set of $n$ points in the plane, each represented by its $x$ and $y$ coordinates. A common operation in
      many geometric algorithms is to sort the points of $S$ around the origin by polar angle, in (say) counterclockwise order from the positive
      $x$-axis. (Assume that ties between points with the same polar angle are broken arbitrarily.) Give an $O(n \log n)$-time algorithm for this
      problem. Your algorithm should not compare points explicitly using their polar angles; instead, it should use cross-products.

      Give a brief description of the main ideas from which the correctness of your method should be evident, augment this with high-level pseudocode as
      in the text, and analyze the running time.
    \end{problem}

    \begin{solution}

      We have the comparison operator ``$\prec$'' such that $\overrightarrow{op} \prec \overrightarrow{oq}$ if the polar angle between the $x$-axis
      and the segment $\overrightarrow{op}$ is less than the angle between the $x$-axis and $\overrightarrow{oq}$. The basic idea for this algorithm
      is that with this comparison operator, we are able to use a merge-sort on the polar angles to sort them in $O(n \log n)$ time.

      Before we sort these angles, we must split the points into points above the $x$-axis and points below the $x$-axis. This is to avoid large
      angles in the counterclockwise direction being counted as small angles in the clockwise direction. After sorting the two subsets, we concatenate
      the sorted list with negative $y$-values to the end of the sorted list with positive $y$-values. This will yield the correct answer if both
      subsets are sorted because all points $p$ with positive $y$-values will form an angle with the positive $x$-axis satisfying $0 < \theta_p < \pi$, and
      all points with negative $y$-values will form an angle with the positive $x$-axis satisfying $\pi < \theta_p < 2 \pi$.

      The above does not explicitly handle the case of points on the $x$-axis. To do so, we group points on the $x$-axis with positive $x$-values with points having
      positive $y$-values, and points on the $x$-axis with negative $x$-values with points having negative $y$-values. It is also important to note
      that the comparison operator $\prec$ allows the polar angles to be sorted without ever actually computing any angle measures.

      These ideas give us the following pseudocode:
      \begin{algorithmic}[1]
        \Function {AngleSort}{$S$}
        \State $S_+ \gets \{ p = (x,y) | y > 0 \}$
        \State $S_- \gets \{ p = (x,y) | y < 0 \}$
        \State $\tilde{S}_+ \gets MergeSort(S_+)$ using the comparison operator $\prec$
        \State $\tilde{S}_- \gets MergeSort(S_-)$ using the comparison operator $\prec$
        \State
        \Return $Concatenate(\tilde{S}_+, \tilde{S}_-)$
        \EndFunction
      \end{algorithmic}

      Splitting $S$ into $S_+$ and $S_-$ is done by scanning $S$, which takes $O(n)$ time. Each of the merge sorts takes $O(n \log n)$ time.
      Therefore, this algorithm runs in $O(n \log n)$ time.

    \end{solution}

  \item
    \begin{problem} The paper below gives elegant, output-sensitive algorithms to compute the convex hull of $n$ points in 2D and 3D, in time $O(n \log h)$, where
      $h$ is the size of the hull. (This bound is optimal.)
      \begin{itemize}
        \item T.M. Chan. Optimal output-sensitive convex hull algorithms in two and three dimensions. \textit{Discrete \& Computational Geometry},
          16:361-368, 1996.
      \end{itemize}

      A link to the paper can be found on the class web page.

      Explain, in your own words, the main ideas behind the 2D algorithm in Section 2.
    \end{problem}

    \begin{solution}
    \end{solution}

  \item
    \begin{problem}
      Problem 2.14, p. 43

      Let $S$ be a set of $n$ disjoint line segments in the plane, and let $p$ be a point not on any of the line segments of $S$. We wish to determine
      all line segments of $S$ that $p$ can see, that is, all line segments of $S$ that contain some point $q$ so that the open segment
      $\overline{pq}$ doesn't intersect any line segment of $S$. Give an $O(n \log n)$ time algorithm for this problem that uses a rotating half-line
      with its endpoint at $p$.
    \end{problem}

    \begin{solution}

      This algorithm will make use of a sweeping line that rotates about the point $p$. We will use an event queue, $Q$, to keep track of when segments are
      to be added and removed. To get segments in the correct order for the sweep line we are using, we use the \texttt{AngleSort} function found in
      problem 1 on the endpoints of the intervals in $S$. We will use a balanced tree, $T$, to keep track of what segments are ``blocking'' each other from $p$'s sight.

      Because the line segments are disjoint, we know that if
      segment $l_1$ ``blocks'' $l_2$ from view as the sweeping line rotates, the next time $l_2$ may be seen is after $l_1$ is removed. Because of
      this, the only event points we need to track are when segments start intersecting the sweep line and when segments stop intersecting the sweep
      line. We can presort the polar angle values of the endpoints of the line segments to fill this event queue before the sweep line begins
      rotating.

      We are going to order the tree $T$ by proximity to $p$. When a line segment $l = \overline{qr}$ which starts at $q$ and travels counterclockwise
      to $r$ is being added to $T$, we can compare its proximity to $p$ relative to other segments in $T$ using a sidedness test. For $l' =
      \overline{st} \in T$, where $\overline{st}$ is traversed in the counterclockwise direction, $l$ is closer to $p$ than $l'$ is if $q$ lies to the
      left of $\overline{st}$. This ``left of'' comparison is how we compare segments when adding them to $T$. By the above observation, two segments
      never switch places in the tree once they have been added.

      As the sweep line rotates, $p$ can see at most one segment at any point in time. Therefore, at each of the event points given by endpoints of
      the line segments, we must simply add the closest segment to $p$ to the list of segments visible to $p$.

      We have the following pseudocode:

      \begin{algorithmic}[1]

        \Function{VisibleSegments}{$S,p$}

        \State $T \gets \emptyset$
        \State $V \gets \emptyset$
        \State Shift coordinates so $p$ is at the origin
        \State $Q \gets \{ \text{ endpoints of segments in } S \}$ where endpoints are associated to the segments they are part of
        \State $Q \gets \texttt{AngleSort}(Q)$ where ties are broken in favor of points closer to the origin (for handling degeneracies)
        \For {$q \in Q$}
        \If { $q$ is a left endpoint }
        \State $T.Insert(q.interval)$
        \ElsIf { $q$ is a right endpoint }
        \State $T.Remove(q.interval)$
        \EndIf
        \State $V \gets V \cup T.minimum$
        \EndFor

        \State
        \Return V

        \EndFunction

      \end{algorithmic}

      Shifting the coordinate system requires adding $-p$ to $p$ and the endpoints of all intervals in $S$, which can be done in $O(n)$ time. As was
      shown in Problem 1, the \texttt{AngleSort} takes $O( n \log n )$ time. Next, we loop over each interval endpoint and either add or remove the
      associated interval to the balanced tree. Each of these insertions and deletions can be done in $O(\log n)$ time. The loop therefore runs in
      $O(n \log n)$ time. Thus, the overall running time of this algorithm is $O(n \log n)$.

      This algorithm needs to be modified to handle a few special cases at this point. First, this algorithm doesn't catch intervals that span the
      positive $x$-axis (test positive $x$-axis for intersection with every interval when initializing $T$). There are also unhandled cases when two
      endpoints (belonging to the same segment or different segments) have the same polar angle.

    \end{solution}
\end{enumerate}
\end{document}
